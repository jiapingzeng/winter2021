\documentclass{article}
\usepackage[margin=0.75in]{geometry}
\usepackage{enumitem}
\usepackage{setspace}
\usepackage{amsmath}
\usepackage{amssymb}
\usepackage{physics}
\usepackage{relsize}
\usepackage{graphicx}
\usepackage{multicol}

\title{CS 143 Midterm}
\date{2/6/2021}
\author{Jiaping Zeng}

\begin{document}
\setstretch{1.35}

\begin{itemize}
      \item [1.]
            \begin{itemize}
                  \item [1.] min: $r$ ($S\subseteq R$), max: $r+s$ (no duplicates)
                  \item [2.] min: $0$ (no common $B$), max: $rs$ (everything joins)
                  \item [3.] min: $0$ (no $S.B$ in $R.B$), max: $r$ (take $R=S$ where $R.B$ are all distinct)
                  \item [4.] min: $0$ ($R.B$ NULL), max: $r$ (no NULL)
                  \item [5.] min: $0$ ($R.B$ all same values), max $r$ (no same $R.B$ values)
            \end{itemize}
      \item [2.]
            \begin{itemize}
                  \item [1.] NO; (a) only gets rid of tuples where both $R.A=S.A$ and $R.B=S.B$, (b) gets rid of tuples where $R.A=S.A$.
                  \item [2.] YES (both are natural joins on $A$)
                  \item [3.] YES
                  \item [4.] YES
                  \item [5.] NO; EXCEPT removes duplicates.
                  \item [6.] NO; (b) counts those that satisfy both conditions twice.
            \end{itemize}
      \item [3.] $\Pi_{R.A,R.B}(\sigma_{R.A=S.A\wedge R.B=S.B}(R\cross S))$
      \item [4.] The result is $\{(1,3),(2,NULL)\}$;\\
            ``FROM $R,S$ WHERE $R.B=S.B$'' gives us $\{(1,1,1,3), (1,1,1,NULL), (1,2,2,NULL)\}$ by matching all rows where $R.B=S.B$, ignoring $NULL$ ones since they cannot be compared so those comparisons will return false. Note: the columns here are in the order of $R.A,R.B,S.B,S.C$.\\
            ``GROUP BY $A,S.B$'' gives us two groups $\{(1,1,1,3), (1,1,1,NULL)\}$ and $\{(1,2,2,NULL)\}$ by grouping rows where $A$ and $S.B$ match.\\
            Finally ``SELECT $S.B,\text{AVG}(C)$'' returns $(1,3)$ and $(2,NULL)$ as AVG ignores null rows when calculating the average of each group.
      \item [5.]
            \begin{itemize}
                  \item [1.] FALSE
                  \item [2.] FALSE
                  \item [3.] TRUE
                  \item [4.] TRUE
                  \item [5.] FALSE
            \end{itemize}
\end{itemize}

\end{document}