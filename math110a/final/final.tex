\documentclass{article}
\usepackage[margin=1in]{geometry}
\usepackage{enumitem}
\usepackage{setspace}
\usepackage{amsmath}
\usepackage{amssymb}
\usepackage{physics}
\usepackage{relsize}
\usepackage{graphicx}
\usepackage{multicol}

\title{Math 110A Final}
\date{3/16/2021}
\author{Jiaping Zeng}

\begin{document}
\setstretch{1.35}

\begin{enumerate}
      \item \begin{enumerate}
                  \item Find the greatest common divisor $(105,133)$ of $105$ and $133$.\\
                        \textbf{Answer}: We can use the Euclidean algorithm to find $(105,133)$ as follows:
                        \[
                              \begin{array}{cc}
                                    133=105\cdot 1+28 & 0\leq 28<105 \\
                                    105=28\cdot 3+21  & 0\leq 21<28  \\
                                    28=21\cdot 1+7    & 0\leq 7<21   \\
                                    21=7\cdot 3+0
                              \end{array}
                        \]
                        Therefore $(105,133)=7$ by the Euclidean algorithm.
                  \item Find a pair of integers $u,v\in\mathbb{Z}$ for which $(105,133)=105u+133v$.\\
                        \textbf{Answer}: Using part (a), we can use backward substitution as follows:
                        \begin{align*}
                              7 & =28-1(105-3\cdot 28)=4\cdot 28-105                 \\
                              7 & =4\cdot (133-1\cdot 105)-105=4\cdot 133-5\cdot 105
                        \end{align*}
                        Therefore $u=-5$ and $v=4$.
            \end{enumerate}
      \item Compute the following remainders:
            \begin{enumerate}
                  \item The remainder when $25^{125}$ is divided by $13$ in $\mathbb{Z}$.\\
                        \textbf{Answer}: Since $25\equiv -1$ (mod $13$), we have \[25^{125}\equiv (-1)^{125}\equiv -1\equiv 12\quad(\text{mod }13)\] by repeatedly applying Theorem 2.2. Therefore the remainder when $25^{125}$ is divided by $13$ is $12$.
                  \item The remainder when $642^{7531}$ is divided by $5$ in $\mathbb{Z}$.\\
                        \textbf{Answer}: By Fermat's Little Theorem (Homework 2 Q4), since $5\nmid 642$, we have $642^{5-1}\equiv 1$ (mod $5$). Then \[642^{7531}\equiv 642^{4^{1882}}\cdot 4^3\equiv 4^3=64\quad(\text{mod }5).\] Therefore the remainder when $642^{7531}$ is divided by $5$ is $64$.
                  \item The remainder when $f(x)=x^{100}-2x^{50}+3x^{15}-4x^2+5$ is divided by $g(x)=x^2-x+1$ in $\mathbb{Q}[x]$.\\
                        \textbf{Answer}: Since $x^3\equiv -1$ (mod $x^2-x+1$), we have
                        \begin{align*}
                              f(x) & =x^{100}-2x^{50}+3x^{15}-4x^2+5                                \\
                                   & =x\cdot x^{3^{33}}-2x^2\cdot x^{3^{16}}+3x\cdot x^{3^5}-4x^2+5 \\
                                   & \equiv x\cdot(-1)^{33}-2x^2\cdot(-1)^{16}+3x\cdot(-1)^5-4x^2+5 \\
                                   & \equiv -x-2x^2-3x-4x^2+5                                       \\
                                   & =-6x^2-4x+5                                                    \\
                                   & \equiv -10x+11\quad(\text{mod }x^2-x+1).
                        \end{align*}
                        Therefore the remainder when $f(x)=x^{100}-2x^{50}+3x^{15}-4x^2+5$ is divided by $g(x)=x^2-x+1$ is $-11x+11$.
            \end{enumerate}
      \item Let $R=\{0_R,a,b,c\}$ be a ring with $4$ elements and additive identity $0_R$. The addition and multiplication tables for $R$ are given below, with some entries missing. Fill in the missing entries. All of the entries in the tables should be one of the symbols `$0_R$', `$a$', `$b$' or `$c$'. Do not give unsimplified answers like `$b+c$'.\\
            \textbf{Answer}:
            \begin{center}
                  \begin{tabular}{c|c c c c}
                        $+$   & $0_R$ & $a$   & $b$   & $c$   \\
                        \hline
                        $0_R$ & $0_R$ & $a$   & $b$   & $c$   \\
                        $a$   & $a$   & $0_R$ & $c$   & $b$   \\
                        $b$   & $b$   & $c$   & $0_R$ & $a$   \\
                        $c$   & $c$   & $b$   & $a$   & $0_R$
                  \end{tabular}\qquad
                  \begin{tabular}{c|c c c c}
                        $\cdot$ & $0_R$ & $a$   & $b$   & $c$   \\
                        \hline
                        $0_R$   & $0_R$ & $0_R$ & $0_R$ & $0_R$ \\
                        $a$     & $0_R$ & $0_R$ & $a$   & $a$   \\
                        $b$     & $0_R$ & $0_R$ & $b$   & $b$   \\
                        $c$     & $0_R$ & $0_R$ & $c$   & $c$
                  \end{tabular}
            \end{center}
      \item In each part, determine whether or not the first ring is isomorphic to the second. Prove that your answer is correct:
            \begin{enumerate}
                  \item $(\mathbb{Z}/2\mathbb{Z})\cross(\mathbb{Z}/2\mathbb{Z})$ and $(\mathbb{Z}/2\mathbb{Z})[x]/(x^2)$.\\
                        \textbf{Answer}: Let $f:(\mathbb{Z}/2\mathbb{Z})\cross(\mathbb{Z}/2\mathbb{Z})\rightarrow(\mathbb{Z}/2\mathbb{Z})[x]/(x^2)$ be defined as $f((a,b))=[ax+b]$. Note that we can also have $f^{-1}([ax+b])=(a,b)$, so $f$ is bijective. Now we have \begin{align*}
                              f((a,b)+(c,d)) & =f((a+c,b+d))      \\
                                             & =[(a+c)x+(b+d)]    \\
                                             & =[ax+b]+[cx+d]     \\
                                             & =f((a,b))+f((c,d))
                        \end{align*}
                        and
                        \begin{align*}
                              f((a,b)(c,d)) & =f((ac,bd))       \\
                                            & =[(ac)x+(bd)]     \\
                                            & \neq[ax+b][cx+d]  \\
                                            & =f((a,b))f((c,d))
                        \end{align*}
                        Therefore $(\mathbb{Z}/2\mathbb{Z})\cross(\mathbb{Z}/2\mathbb{Z})$ is not isomorphic to $(\mathbb{Z}/2\mathbb{Z})[x]/(x^2)$.
                  \item $\mathbb{Q}[x]/(x^2+1)$ and $\mathbb{Q}[x]/(x^2+4)$.\\
                        \textbf{Answer}: Let $f:\mathbb{Q}[x]/(x^2+1)\rightarrow\mathbb{Q}[x]/(x^2+4)$ be defined as $f([ax+b])=[ax+b]\in\mathbb{Q}[x]/(x^2+4)$. Note that we can also have $f^{-1}([ax+b])=[ax+b]\in\mathbb{Q}[x]/(x^2+1)$, so $f$ is bijective. Now we have \begin{align*}
                              f([ax+b]+[cx+d]) & =f([(a+c)x+(b+d)])  \\
                                               & =[(a+c)x+(b+d)]     \\
                                               & =[ax+b]+[cx+d]      \\
                                               & =f([ax+b])+([cx+d])
                        \end{align*}
                        and
                        \begin{align*}
                              f([ax+b][cx+d]) & =f([(ad+bc)x+bd])  \\
                                              & =[(ad+bc)x+bd]     \\
                                              & =[ax+b][cx+d]      \\
                                              & =f([ax+b])([cx+d])
                        \end{align*}
                        Therefore $\mathbb{Q}[x]/(x^2+1)$ is isomorphic to $\mathbb{Q}[x]/(x^2+4)$.
            \end{enumerate}
      \item Which of the following polynomials are irreducible in the given polynomial rings? Prove that your answer is correct.
            \begin{enumerate}
                  \item $f(x)=x^{10}+\pi x^8+3x^3+2x^2+\sqrt{110}$ in $\mathbb{R}[x]$.\\
                        \textbf{Answer}: $f(x)$ is not irreducible in $\mathbb{R}[x]$ by Theorem 4.30.
                  \item $f(x)=x^5+x^4+1$ in $(\mathbb{Z}/2\mathbb{Z})[x]$.\\
                        \textbf{Answer}: Neither $0$ nor $1$ is a root of $f(x)$, so $f(x)$ can only factor into the product of a quadratic polynomial and a cubic polynomial. Therefore one of $x^2,x^2+1,x^2+x,x^2+x+1$ must be a factor.
                        \begin{center}
                              \begin{tabular}{|c|c|c|}
                                    \hline
                                    $g(x)$    & is factor? & $f(x)/g(x)$ \\
                                    \hline
                                    $x^2$     & no         & -           \\
                                    \hline
                                    $x^2+1$   & no         & -           \\
                                    \hline
                                    $x^2+x$   & no         & -           \\
                                    \hline
                                    $x^2+x+1$ & yes        & $x^3+x+1$   \\
                                    \hline
                              \end{tabular}
                        \end{center}
                        Therefore $x^5+x^4+1=(x^2+x+1)(x^3+x+1)$ so $f(x)$ is not irreducible.
                  \item $f(x)=x^4+3x^3+5x+1$ in $\mathbb{Q}[x]$.\\
                        \textbf{Answer}: $f(x)$ is irreducible; by contradiction: suppose $f(x)$ is reducible, then it can be factored as the product of two nonconstant polynomials in $\mathbb{Q}[x]$. If either of those factors has degree $1$, then $f(x)$ has a root in $\mathbb{Q}$. But the Rational Root Test shows that $f(x)$ has no roots in $\mathbb{Q}$ (the only possibilities are $\pm 1$ and neither is a root). Thus if $f(x)$ is reducible, the only possible factorization is as a product of two quadratics, by Theorem 4.2. In this case Theorem 4.23 shows that there is such a factorization in $\mathbb{Z}[x]$. Furthermore, there is a factorization as a product of monic quadratics in $\mathbb{Z}[x]$, i.e. \[(x^2+ax+b)(x^2+cx+d)=x^4+3x^3+5x+1,\] with $a,b,c,d\in\mathbb{Z}$. Multiplying out the left-hand side, we have \[x^4+(a+c)x^3+(ac+b+d)x^2+(ad+bc)x+bd=x^4+3x^3+0x^2+5x+1.\] Equal polynomials have equal coefficients; hence, \[a+c=3,ac+b+d=0,ad+bc=5,bd=1.\] Since $bd=1\in\mathbb{Z}$ implies that $b=d=1$ or $b=d=-1$, using the third equation we have two possibilities: $ad+bc=5\implies a+c=\pm 5$. But this contradicts with the first equation, so a factorization of $f(x)$ as a product of quadratics in $\mathbb{Z}[x]$, and, hence in $\mathbb{Q}[x]$, is impossible. Therefore, $f(x)$ is irreducible in $\mathbb{Q}[x]$.
            \end{enumerate}
      \item Give an example of a field of order $125$ (that is, a finite field containing \textit{exactly} 125 elements). Note that $125=5^3$.\\
            \textbf{Answer}: $(\mathbb{Z}/5\mathbb{Z}[x])/(x^3+x^2+1)$ contains exactly 125 elements. Take $x^3+x^2+1$ in $\mathbb{Z}/5\mathbb{Z}[x]$, it is irreducible since the only possible roots are $\pm 1$ and neither is a root. So the possible remainders on division by $x^3+x^2+1$ in $\mathbb{Z}/5\mathbb{Z}[x]$ are the polynomials of the form $a_0+a_1x+a_2^2$, with $a_k\in\mathbb{Z}/5\mathbb{Z}$. There are $5$ possibilities for each of the $3$ coefficients, so there are $5^3$ different polynomials of this form. Consequently, by Corollary 5.5, there are exactly $5^3=125$ distinct congruence classes in $(\mathbb{Z}/5\mathbb{Z}[x])/(x^3+x^2+1)$.
      \item Let $I=\bigg\{a_0+a_1x+\cdots+a_nx^n\in\mathbb{Z}[x]\bigg| 5|a_0,5|a_1\bigg\}\subseteq\mathbb{Z}[x]$ be the set of all polynomials with integer coefficients with constant \textit{and} linear terms divisible by $5$.
            \begin{enumerate}
                  \item Prove that $I$ is an ideal in $\mathbb{Z}[x]$.\\
                        \textbf{Answer}: Take $p(x)=a_0+a_1x+\cdots+a_nx^n\in I$ and $q(x)=b_0+b_1x+\cdots+b_nx^n\in I$ where $5|a_0,5|a_1,5|b_0,5|b_1$. Then by definition of divisibility there must exist $m_0,m_1$ and $n_0,n_1$ such that $a_0=5m_0,a_1=5m_1,b_0=5n_0,b_1=5n_1$. Then \begin{align*}
                              p(x)-q(x) & =a_0+a_1x+\cdots+a_nx^n-b_0-b_1x-\cdots-b_nx^n     \\
                                        & =5m_0+5m_1x+\cdots+a_nx^n-5n_0-5n_1x-\cdots-b_nx^n \\
                                        & =5(m_0-n_0)+5(m_1-n_1)x+\cdots+(a_n-b_n)x^n.
                        \end{align*}
                        Since $5(m_0-n_0)$ and $5(m_1-n_1)$ are clearly divisible by $5$, $p(x)-q(x)\in I$. Now take any $r(x)=r_0+r_1x+\cdots+r_nx^n\in\mathbb{Z}[x]$, we have \begin{align*}
                              p(x)r(x) & =a_0r_0+(a_0r_1+a_1r_0)x+\cdots    \\
                                       & =5m_0r_0+(5m_0r_1+5m_1r_0)x+\cdots \\
                                       & =5m_0r_0+5(m_0r_1+m_1r_0)x+\cdots
                        \end{align*}
                        and \begin{align*}
                              r(x)p(x) & =r_0a_0+(r_1a_0+r_0a_1)x+\cdots    \\
                                       & =5r_0m_0+(5r_1m_0+5r_0m_1)x+\cdots \\
                                       & =5m_0r_0+5(m_0r_1+m_1r_0)x+\cdots.
                        \end{align*}
                        Since $5m_0r_0$ and $5(m_0r_1+m_1r_0)$ are divisible by $5$, we have $p(x)r(x)\in I$ and $r(x)p(x)\in I$. Therefore $I$ is an ideal in $\mathbb{Z}[x]$ by Theorem 6.1.
                  \item Find two polynomials $f(x),g(x)\in\mathbb{Z}$ which generate $I$ (i.e. $I=(f(x),g(x))$). Prove that they generate $I$.\\
                        \textbf{Answer}: Let $f(x)=5$ and $g(x)=x^2$, we will show that $I=(f(x),g(x))$. Since $5|a_0$ and $5|a_1$, we can take $m_0,m_1\in\mathbb{Z}$ such that $a_0=5m_0$ and $a_1=5m_1$. Then we have \begin{align*}
                              a_0+a_1x+\cdots+a_nx^n & =5(m_0+m_1x)+x^2(a_2+a_3x+\cdots+a_nx^{n-2})          \\
                                                     & =f(x)(m_0+m_1x)+g(x)(a_2+a_3x+\cdots+a_nx^{n-2})\in I
                        \end{align*}
                        by Theorem 6.1 since $m_0+m_1x$ and $a_2+a_3x+\cdots+a_nx^{n-2}$ are both in $\mathbb{Z}[x]$, so $I=(f(x),g(x))$.
                  \item Prove that $I$ is not a principal ideal in $\mathbb{Z}[x]$ (i.e. $I$ cannot be written in the form $I=(h(x))$ for any polynomial $h(x)\in\mathbb{Z}[x]$).\\
                        \textbf{Answer}: By contradiction. Suppose that there is an $h(x)$ such that $I=(h(x))$, then $h(x)$ must be a constant or else it won't be able to generate the constant term in polynomials in $I$ with a nonzero constant coefficient, i.e. we must have $h(x)=h_0$ for some $h_0\in\mathbb{Z}$. Since $h(x)\in I$, we must also have $5|h_0$ as $h_0$ is a constant coefficient. However now we cannot generate polynomials where the $x^2$ or higher terms have coefficients that are not divisible by $5$ (e.g. we cannot generate $x^2$ which should be in $I$). Therefore there is no such $h(x)$ and $I$ is not a principal ideal in $\mathbb{Z}$.
            \end{enumerate}
      \item Let $R$ and $S$ be rings. If $I\subseteq R$ and $J\subseteq S$ are ideals in $R$ and $S$, respectively, let \[I\cross J=\{(x,y)\mid x\in I,y\in J\}\subseteq R\cross S\]
            \begin{enumerate}
                  \item If $I$ and $J$ are any ideals in $R$ and $S$, prove that $I\cross J$ is an ideal in $R\cross S$ and \[(R\cross S)/(I\cross J)\cong(R/I)\cross(S/J).\]
                        \textbf{Answer}: Take $(x_1,y_1),(x_2,y_2)\in I\cross J$, then \[(x_1,y_1)-(x_2,y_2)=(x_1-x_2,y_1-y_2)\in I\cross J\] since $x_1-x_2\in I$ and $y_1-y_2\in J$ by Theorem 6.1 as $x_1,x_2\in I$ and $y_1,y_2\in J$. Now take $(r,s)\in R\cross S$, we have \[(r,s)(x,y)=(rx,sy)\] and \[(x,y)(r,s)=(xr,ys).\] Note that since $x\in I$, by Theorem 6.1 $rx$ and $xr$ are also in $I$. Similarly, since $y\in J$, $sy$ and $ys$ are also in $J$. Therefore both $(rx,sy)$ and $(xr,ys)$ are in $I\cross J$, so $I\cross J$ is an ideal by Theorem 6.1.
                  \item Now assume that $R$ and $S$ have (multiplicative) identities $1_R$ and $1_S$. If $K\subseteq R\cross S$ is any ideal of $R\cross S$, prove that there are ideals $I\subseteq R$ and $J\subseteq S$ for which $K=I\cross J$.\\
                        \textbf{Answer}: Take $I=\{r\in R\mid (r,0_R)\in K\}$ and $J=\{s\in S\mid (0_R,s)\in K\}$, clearly we have $I\subseteq R$ and $J\subseteq S$ by definition of $K\subseteq R\cross S$, as well as $I\cross J\subseteq K$. Now take $r\in I$ and $s\in J$; since $R$ and $S$ have multiplicative identities $1_R$ and $1_S$, we have \[r(1_R,0_S)+s(0_R,1_S)=(r,0_S)+(0_R,s)=(r,s)\] for every $(r,s)\in K$, so $K\subseteq I\cross J$. Therefore $K=I\cross J$.\\
                        Now we will show that $I$ and $J$ are ideals; take $(r_1,0_S),(r_2,0_S)\in K$, by Theorem 6.1 we have \[(r_1,0_R)-(r_2,0_R)=(r_1-r_2,0_R)\in K,\] so $r_1-r_2\in I$. Similarly for $(0_R,s_1),(0_R,s_2)\in K$ we have $s_1-s_2\in J$. In addition, by Theorem 6.1 we also have \[(r,s)(r_1,0_S)=(rr_1,0_S)\in K\] and \[(r_1,0_S)(r,s)=(r_1r,0_S)\in K\] for $(r,s)\in R\cross S$, so $rr_1,r_1r\in I$ and similarly $ss_1,s_1s\in J$. Therefore $I,J$ are both ideals by Theorem 6.1.
            \end{enumerate}
\end{enumerate}
\newpage
I assert, on my honor, that I have not received assistance of any kind from any other person, or given assistance to any other person, while working on the midterm.\\
Signature: \includegraphics[width=2in]{signature.png}

\end{document}