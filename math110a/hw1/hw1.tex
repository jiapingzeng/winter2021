\documentclass{article}
\usepackage[margin=1in]{geometry}
\usepackage{enumitem}
\usepackage{setspace}
\usepackage{amsmath}
\usepackage{amssymb}
\usepackage{physics}
\usepackage{relsize}

\title{Math 110A Homework 1}
\date{1/9/2020}
\author{Jiaping Zeng}

\begin{document}
\setstretch{1.35}
\maketitle

\begin{enumerate}
      \item Let $a$ be any integer and let $b$ and $c$ be positive integers. Suppose that when $a$ is divided by $b$, the quotient is $q$, and the remainder is $r$, so that $a=bq+r$ and $0\leq r<b$. If $ac$ is divided by $bc$, show that the quotient is $q$ and the remainder is $rc$.\\
            \textbf{Answer}: Since $c$ is a positive integer, we have $a=bq+r\implies ac=(bq+r)c\implies ac=(bc)q+rc$ and $0\leq r<b\implies 0\leq rc<bc$. Then by Division Algorithm, the quotient is $bc$ and the remainder is $rc$.
      \item Let $n$ be a positive integer. Prove that $a$ and $c$ leave the same remainder when divided by $n$ if and only if $a-c=nk$ for some integer $k$.\\
            \textbf{Answer}:\begin{itemize}
                  \item [$\Rightarrow$:] Suppose $a$ and $c$ leave the same remainder when divided by $n$, we want to show that $a-c=nk$ for some integer $k$. By Division Algorithm, we have $a=np+r$ and $c=nq+r$ for quotients $p,q\in\mathbb{Z}$ and remainder $r\in\mathbb{Z}$. Then $a-c=np+r-nq-r=n(p-q)$. Since $p$ and $q$ are both integers, so is their difference. So we can let $k=p-q$ and we have $a-c=nk$ for $k\in\mathbb{Z}$.
                  \item [$\Leftarrow$:] Suppose $a-c=nk$ for some integer $k$, we want to show that $a$ and $c$ leave the same remainder when divided by $n$. By Division Algorithm, we have $a=np+r$ and $c=nq+s$ for quotients $p,q\in\mathbb{Z}$ and remainders $r,s\in\mathbb{Z}$ with $0\leq r<n$ and $0\leq s<n$. Then by substitution we have $a-c=np+r-nq-s=n(p-q)+(r-s)\implies r-s=(a-c)-n(p-q)\implies r-s=nk-n(p-q)=n(k-p+q)$, i.e. $n$ divides $r-s$. However, since we have $0\leq r<n$ and $0\leq s<n$, which implies that $0\leq r-s<n$, $n$ can only divide $r-s$ if $r-s=0$, i.e. $r=s$. Therefore $a$ and $c$ leave the same remainder when divided by $n$.
            \end{itemize}
      \item Suppose $a,b,q$ and $r$ are integers such that $a=bq+r$. Prove the following:
            \begin{enumerate}
                  \item Every common divisor $c$ of $a$ and $b$ is also a common divisor of $b$ and $r$.\\
                        \textbf{Answer}: Since $c$ divides both $a$ and $b$, we have $a=cs$ and $b=ct$ for some $s,t\in\mathbb{Z}$. Then by substitution we have $a=bq+r\implies cs=ctq+r\implies r=cs-ctq=c(s-tq)$. Therefore $c$ also divides $r$ and is a common divisor of $b$ and $r$.
                  \item Every common divisor of $b$ and $r$ is also a common divisor of $a$ and $b$.\\
                        \textbf{Answer}: Let $m$ be an arbitrary common divisor of $b$ and $r$, then $b=mj$ and $r=mk$ for some $j,k\in\mathbb{Z}$. Then by substitution we have $a=bq+r\implies a=mjq+mk=m(jq+k)$. Therefore $m$ also divides $a$ and is a common divisor of $a$ and $b$.
                  \item $(a,b)=(b,r)$.\\
                        \textbf{Answer}: By parts (a) and (b), every common divisor of $a$ and $b$ is a common divisor of $b$ and $r$, and every common divisor of $b$ and $r$ is a common divisor of $a$ and $b$. Therefore $a,b$ and $b,r$ share the same common divisors and must therefore have the same greatest common divisor, i.e. $(a,b)=(b,r)$.
            \end{enumerate}
      \item Use the Euclidean algorithm (see Exercise 1.2.15) to compute the gcd (123,90), and find integers $u$ and $v$ with $(123,90)=123u+90v$. Show your work.\\
            \textbf{Answer}: Using the Euclidean algorithm, we have the following:\\
            $123=90\cdot 1+33, 0\leq 33<90$\\
            $90=33\cdot 2+24, 0\leq 66<90$\\
            $33=24\cdot 1+9, 0\leq 24<33$\\
            $24=9\cdot 2+6, 0\leq 6<9$\\
            $9=6\cdot 1+3, 0\leq 3<6$\\
            $6=3\cdot 2+0$\\
            Therefore $(123,90)=3$.
      \item \begin{enumerate}
                  \item If $(a,c)=1$ and $(b,c)=1$, prove that $(ab,c)=1$.\\
                        \textbf{Answer}: Since $(a,c)=1$, we must have $au_1+cv_1=1$ for some $u_1,v_1\in\mathbb{Z}$. Similarly, we also must have $bu_2+cv_2=1$ for some $u_2,v_2\in\mathbb{Z}$. Upon multiplying the two equations, we have $(au_1+cv_1)(bu_2+cv_2)=1\implies abu_1u_2+acu_1v_2+bcu_2v_1+c^2v_1v_2=1\implies ab(u_1u_2)+c(au_1v_2+bu_2v_1+cv_1v_2)=1$. Now suppose $(ab,c)=d$, then we must have $ab=dm$ and $c=dn$ for some $m,n\in\mathbb{Z}$. By substitution we have $dm(u_1u_2)+dn(au_1v_2+bu_2v_1+cv_1v_2)=1\implies d|1$. Therefore $d=(ab,c)=1$.
                  \item Use induction and part (a) to show that if $(a,b)=1$ then $(a,b^n)=1$ for all integers $n\geq 1$.\\
                        \textbf{Answer}: By induction on $n$:\\
                        Base case: $n=1$; we want to show that $(a,b)=1$, which is true by our assumption.\\
                        Inductive step: Suppose that $(a,b)=1\implies (a,b^n)=1$, we want to show that $(a,b^{n+1})=1$ also. First we note that $(m,n)=(n,m)$ trivially, which lets us swap the variables when using part (a). Now we apply part (a) which gives us $(a,b^n\cdot b)=1\implies (a,b^{n+1})=1$.\\
                        Therefore $(a,b)=1\implies (a,b^n)=1$ by induction.
            \end{enumerate}
      \item Let $a,b,c\in\mathbb{Z}$. Prove that the equation $ax+by=c$ has integer solutions if and only if $(a,b)|c$.\\
            \textbf{Answer}: Let $d=(a,b)$, then we must have $au+bv=d$ for some $u,v\in\mathbb{Z}$. In addition, since $d|a$ and $d|b$, we also have $a=dm$ and $b=dn$ for some $m,n\in\mathbb{Z}$.
            \begin{itemize}
                  \item [$\Rightarrow$:] Suppose $ax+by=c$ has integer solutions, i.e. $x,y\in\mathbb{Z}$, we want to show that $d|c$. By substitution we have $dmx+dny=c\implies d(mx+ny)=c$, which implies that $d|c$.
                  \item [$\Leftarrow$:] Suppose that $d|c$, we want to show that $ax+by=c$ has integer solutions. Since $d|c$, there must exist some $k\in\mathbb{Z}$ such that $c=dk$. By substitution we have $ax+by=dk\implies ax+by=(au+bv)k\implies ax+by=auk+bvk$. Then we can take $x=uk$ and $y=vk$; since $u,v,k$ are all integers, so are $x$ and $y$.
            \end{itemize}
      \item Suppose that $a=p_1^{r_1}p_2^{r_2}\cdots p_k^{r_k}$ where $p_1,p_2,\ldots,p_k$ are distinct positive primes and each $r_i\geq 0$. Find a formula for the number of positive divisors of $a$, in terms of the exponents $r_i$.\\
            \textbf{Answer}: To construct a positive divisor, we can "choose" an exponent for each $p_i$ and multiply the result together. Note that we can choose from 0 to $r_i$ for each $p_i$, giving us $r_i+1$ choices. Therefore, we have $\prod_k (r_i+1)=k\prod_k r_i$ possible positive divisors.
      \item For any integer $n>0$, prove that $a|b$ if and only if $a^n|b^n$.\\
            \textbf{Answer}:
            \begin{itemize}
                  \item [$\Rightarrow$:] Suppose that $a|b$, we want to show that $a^n|b^n$. Since $a|b$, there must exist some $m\in\mathbb{Z}$ such that $b=ma$. Since $n>0$, we have $b^n=(ma)^n\implies b^n=m^na^n$, where $m^n\in\mathbb{Z}$. Therefore $a^n|b^n$.
                  \item [$\Leftarrow$:] Suppose that $a^n|b^n$, we want to show that $a|b$. By prime factorization, we have $a=p_1^{r_1}p_2^{r_2}\cdots p_k^{r_k}$ and $b=p_1^{s_1}p_2^{s_2}\cdots p_k^{s_k}$, where $p_1,p_2,\ldots,p_k$ are distinct positive primes and each $r_i,s_i\geq 0$. Then we also have $a^n=p_1^{nr_1}p_2^{nr_2}\cdots p_k^{nr_k}$ and $b^n=p_1^{ns_1}p_2^{ns_2}\cdots p_k^{ns_k}$ by substitution. Since $a^n|b^n$, we must have $ns_i\geq nr_i$ for each $i$; then since $n\geq 1$, we also have $s_i\geq r_i$ for each $i$, i.e. $s_i=r_i+t_i$ where $t_i\geq 0$. Again by substitution we have $b=p_1^{s_1}p_2^{s_2}\cdots p_k^{s_k}=(p_1^{t_1}p_2^{t_2}\cdots p_k^{t_k})\cdot(p_1^{r_1}p_2^{r_2}\cdots p_k^{r_k})=(p_1^{t_1}p_2^{t_2}\cdots p_k^{t_k})a$. Since $(p_1^{t_1}p_2^{t_2}\cdots p_k^{t_k})\in\mathbb{Z}$, $a$ divides $b$ by definition.
            \end{itemize}
      \item For any integers $m$ and $n$ with $0\leq m\leq n$, let $\mathlarger{\binom{n}{m}=\frac{n!}{m!(n-m)!}}$. Recall that these are the \textit{binomial coefficients} in the binomial theorem: \[(a+b)^n=\mathlarger{\sum_{m=0}^n\binom{n}{m}}a^mb^{n-m}.\] It is know that $\mathlarger{\binom{n}{m}}$ is an integer. Let $p$ be a prime and let $k$ be an integer with $1\leq k\leq p-1$. Prove that $p$ divides $\mathlarger{\binom{p}{k}}$.\\
            \textbf{Answer}: We have $\mathlarger{\binom{p}{k}}=\dfrac{p!}{k!(p-k)!}=p\cdot\dfrac{(p-1)!}{k!(p-k)!}$. Since $p$ is prime and the denominator is the product of integers strictly less than $p$, by prime factorization there is no prime factor in $k!(p-k)!$ that divides $p$. Then for $\mathlarger{\binom{p}{k}}$ to be an integer as given, we must have $k!(p-k)!|(p-1)!$, i.e. $\dfrac{(p-1)!}{k!(p-k)!}\in\mathbb{Z}$. Therefore $p$ divides $\mathlarger{\binom{p}{k}}$ by definition.
\end{enumerate}
\end{document}