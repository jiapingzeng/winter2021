\documentclass{article}
\usepackage[margin=1in]{geometry}
\usepackage{enumitem}
\usepackage{setspace}
\usepackage{amsmath}
\usepackage{amssymb}
\usepackage{physics}
\usepackage{relsize}

\title{Math 110A Homework 2}
\date{1/17/2021}
\author{Jiaping Zeng}

\begin{document}
\setstretch{1.35}
\maketitle

\begin{enumerate}
      \item
            \begin{enumerate}
                  \item Which of $[0],[1],[2],[3]$ is equal to $[5^{2000}]$ in $\mathbb{Z}/4\mathbb{Z}$?\\
                        \textbf{Answer}: Since $5\equiv 1$ (mod 4), by repeatedly applying Theorem 2.2 we have $5^{2000}\equiv 1^{2000}=1$ (mod 4). Then by Theorem 2.3 $[5^{2000}]\equiv [1]$ (mod 4).
                  \item Which of $[0],[1],[2],[3],[4]$ is equal to $[4^{2001}]$ in $\mathbb{Z}/5\mathbb{Z}$?\\
                        \textbf{Answer}: Since $4\equiv -1$ (mod 5), by repeatedly applying Theorem 2.2 we have $4^{2001}\equiv (-1)^{2001}=-1$ (mod 5). Then by Theorem 2.3 $[4^{2001}]\equiv[-1]\equiv[4]$ (mod 5).
            \end{enumerate}
      \item If $a\in\mathbb{Z}$, prove that $a^2$ is not congruent to $2$ or $3$ modulo $4$.\\
            \textbf{Answer}: By Corollary 2.5, $a$ must be congruent to one of $0,1,2,3$ (mod 4). Then by Theorem 2.2, $a^2$ must be congruent to one of $0^2,1^2,2^2,3^2$. Note that $2^2=4\equiv 0$ (mod 4) and $3^2=9\equiv 1$ (mod 4), therefore $a^2$ can only be congruent to $0$ or $1$.
      \item
            \begin{enumerate}
                  \item Prove or disprove: If $a^2\equiv b^2$ (mod $n$), then $a\equiv b$ (mod $n$) or $a\equiv -b$ (mod $n$).\\
                        \textbf{Answer}: Disprove by counter example: let $a=2$ and $b=4$, then $a^2=4\equiv 16=b^2$ (mod 4). However, $2\not\equiv 4$ (mod 4) and $2\not\equiv -4$ (mod 4).
                  \item Do part (a) when $n$ is prime.\\
                        \textbf{Answer}: By definition, $a^2\equiv b^2$ (mod n) implies that $n$ divides $a^2-b^2$. By difference of squares, $n$ divides $(a+b)(a-b)$, then since $n$ is prime it must divide either $a+b$ or $a-b$ by Theorem 1.5. If $n$ divides $a+b$, $a\equiv -b$ (mod $n$) by definition and if $n$ divides $a-b$, $a\equiv b$ (mod $n$).
            \end{enumerate}
      \item \textit{Fermat's Little Theorem}. Let $p$ be a positive prime number.
            \begin{enumerate}
                  \item Prove that for any $a,b\in\mathbb{Z}$, $(a+b)^p\equiv a^p+b^p$ (mod $p$).\\
                        \textbf{Answer}: By binomial theorem, $(a+b)^p=\sum_{m=0}^p\binom{p}{m}a^mb^{p-m}$. By exercise 1.3.25, $p$ divides $\binom{p}{k}$ for $1\leq k\leq p-1$, i.e. $p$ divides every term in the polynomial except the two with coefficients $\binom{p}{0}$ and $\binom{p}{p}$, which corresponds to the terms $\binom{p}{0}b^p=b^p$ and $\binom{p}{p}a^p=a^p$. Since the other terms are divisible by $p$, by Theorem 2.3 they are congruent to 0 (mod $p$). Therefore $(a+b)^p\equiv a^p+b^p$.
                  \item Prove by induction that $a^p\equiv a$ (mod $p$) for all nonnegative integers $a$.\\
                        \textbf{Answer}: By induction on $a$.\\
                        Base case: $a=1$, then clearly $1^p=1$ is true for any $p$.\\
                        Induction step: suppose $a^p\equiv a$ (mod $p$), we want to show that $(a+1)^p\equiv a+1$ (mod $p$). This is trivial upon substituting $b=1$ into part (a).\\
                        Therefore $a^p\equiv a$ (mod $p$) for $a\geq 1$.
                  \item Prove that if $p\nmid a$, then $a^{p-1}\equiv 1$ (mod $p$).\\
                        \textbf{Answer}: Since $p\nmid a\implies (a,p)=1$, by Theorem 2.10 $a$ is a unit in $\mathbb{Z}/p\mathbb{Z}$ with an inverse $b$ such that $ab=1$. By part (b), $a\equiv a^p$ (mod $p$), so $ab=1\implies a^pb\equiv 1\implies a^{p-1}(ab)\equiv 1\implies a^{p-1}\equiv 1$ (mod $p$).
                  \item Find the remainder when $3^{1000}$ is divided by $7$, without using a calculator or computer.\\
                        \textbf{Answer}: Since $7\nmid 3$, we know that $3^6\equiv 1$ (mod 7) by part (c). Then $3^{1000}=3^{6^{166}}\cdot 3^4\equiv 3^4=81\equiv 4$ (mod 7). Therefore the remainder is 4.
            \end{enumerate}
      \item Write out the addition and multiplication tables for
            \begin{enumerate}
                  \item $\mathbb{Z}/4\mathbb{Z}$\\
                        \textbf{Answer}:
                        \begin{center}
                              \begin{tabular}{c|c c c c}
                                    $\oplus$ & $[0]$ & $[1]$ & $[2]$ & $[3]$ \\
                                    \hline
                                    $[0]$    & 0     & 1     & 2     & 3     \\
                                    $[1]$    & 1     & 2     & 3     & 0     \\
                                    $[2]$    & 2     & 3     & 0     & 1     \\
                                    $[3]$    & 3     & 0     & 1     & 2
                              \end{tabular}
                              \begin{tabular}{c|c c c c}
                                    $\odot$ & $[0]$ & $[1]$ & $[2]$ & $[3]$ \\
                                    \hline
                                    $[0]$   & 0     & 0     & 0     & 0     \\
                                    $[1]$   & 0     & 1     & 2     & 3     \\
                                    $[2]$   & 0     & 2     & 0     & 2     \\
                                    $[3]$   & 0     & 3     & 2     & 1
                              \end{tabular}
                        \end{center}
                  \item $\mathbb{Z}/7\mathbb{Z}$\\
                        \textbf{Answer}:
                        \begin{center}
                              \begin{tabular}{c|c c c c c c c}
                                    $\oplus$ & $[0]$ & $[1]$ & $[2]$ & $[3]$ & $[4]$ & $[5]$ & $[6]$ \\
                                    \hline
                                    $[0]$    & 0     & 1     & 2     & 3     & 4     & 5     & 6     \\
                                    $[1]$    & 1     & 2     & 3     & 4     & 5     & 6     & 0     \\
                                    $[2]$    & 2     & 3     & 4     & 5     & 6     & 0     & 1     \\
                                    $[3]$    & 3     & 4     & 5     & 6     & 0     & 1     & 2     \\
                                    $[4]$    & 4     & 5     & 6     & 0     & 1     & 2     & 3     \\
                                    $[5]$    & 5     & 6     & 0     & 1     & 2     & 3     & 4     \\
                                    $[6]$    & 6     & 0     & 1     & 2     & 3     & 4     & 5
                              \end{tabular}
                              \begin{tabular}{c|c c c c c c c}
                                    $\odot$ & $[0]$ & $[1]$ & $[2]$ & $[3]$ & $[4]$ & $[5]$ & $[6]$ \\
                                    \hline
                                    $[0]$   & 0     & 0     & 0     & 0     & 0     & 0     & 0     \\
                                    $[1]$   & 0     & 1     & 2     & 3     & 4     & 5     & 6     \\
                                    $[2]$   & 0     & 2     & 4     & 6     & 1     & 3     & 5     \\
                                    $[3]$   & 0     & 3     & 6     & 2     & 5     & 1     & 4     \\
                                    $[4]$   & 0     & 4     & 1     & 5     & 2     & 6     & 3     \\
                                    $[5]$   & 0     & 5     & 3     & 1     & 6     & 4     & 2     \\
                                    $[6]$   & 0     & 6     & 5     & 4     & 3     & 2     & 1
                              \end{tabular}
                        \end{center}
            \end{enumerate}
      \item Solve the equation $x^2\oplus [3]\odot x\oplus [2]=[0]$ in $\mathbb{Z}/6\mathbb{Z}$.\\
            \textbf{Answer}:
            \begin{center}
                  \begin{tabular}{c|c|c}
                        $x$   & $x^2\oplus [3]\odot x\oplus [2]$                        & is solution? \\
                        \hline
                        $[0]$ & $[0]\odot [0]\oplus 3\odot [0]\oplus 2=[0]+[0]+[2]=[2]$ & yes          \\
                        $[1]$ & $[1]\odot [1]\oplus 3\odot [1]\oplus 2=[1]+[3]+[2]=[0]$ & no           \\
                        $[2]$ & $[2]\odot [2]\oplus 3\odot [2]\oplus 2=[4]+[0]+[2]=[0]$ & no           \\
                        $[3]$ & $[3]\odot [3]\oplus 3\odot [3]\oplus 2=[3]+[3]+[2]=[2]$ & yes          \\
                        $[4]$ & $[4]\odot [4]\oplus 3\odot [4]\oplus 2=[2]+[0]+[2]=[4]$ & no           \\
                        $[5]$ & $[5]\odot [5]\oplus 3\odot [5]\oplus 2=[1]+[3]+[2]=[0]$ & no
                  \end{tabular}
            \end{center}
            Therefore the equation has two solutions: $[0]$ and $[2]$.
      \item
            \begin{enumerate}
                  \item Find an element $[a]$ in $\mathbb{Z}/7\mathbb{Z}$ such that every nonzero element of $\mathbb{Z}/7\mathbb{Z}$ is a power of $[a]$.\\
                        \textbf{Answer}: Let $a=3$, then we have $[a]=[3],[a]^2=[2],[a]^3=[6],[a]^4=[4],[a]^5=[5],[a]^6=[1]$ as desired.
                  \item Do $(a)$ in $\mathbb{Z}/5\mathbb{Z}$.\\
                        \textbf{Answer}: Again let $a=3$, then we have $[a]=[3],[a]^2=[4],[a]^3=[2],[a]^4=[1]$ as desired.
                  \item Can you do (a) in $\mathbb{Z}/6\mathbb{Z}$?\\
                        \textbf{Answer}: No. If we pick an odd $a$ then $a^n$ would also be odd and $[a]^n$ would not contain the even classes (since $a^n-6k$ would be odd for any $k$); similarly if we pick an even $a$ then all powers would be even and $[a]^n$ would not contain the odd classes.
            \end{enumerate}
      \item Find all units and zero divisors in
            \begin{enumerate}
                  \item $\mathbb{Z}/8\mathbb{Z}$\\
                        \textbf{Answer}: $1,3,5,7$ are units in $\mathbb{Z}/8\mathbb{Z}$ because $3\cdot 3=1$, $5\cdot 5=1$ and $7\cdot 7=1$; $2,4$ are zero divisors because $2\cdot 4=0$.
                  \item $\mathbb{Z}/9\mathbb{Z}$\\
                        \textbf{Answer}: $1,2,4,5,7,8$ are units in $\mathbb{Z}/9\mathbb{Z}$ because $2\cdot 5=1$, $4\cdot 7=1$ and $8\cdot 8=1$; $3$ is a zero divisor because $3\cdot 3=0$.
                  \item $\mathbb{Z}/10\mathbb{Z}$\\
                        \textbf{Answer}: $1,3,7,9$ are units in $\mathbb{Z}/10\mathbb{Z}$ because $3\cdot 7=1$ and $9\cdot 9=1$; $2,5$ are zero divisors because $2\cdot 5=0$.
            \end{enumerate}
      \item Let $a,b,n$ be integers with $n>1$. Let $d=(a,n)$ and assume that $d|b$. Prove that the equation $[a]x=[b]$ has exactly $d$ solutions in $\mathbb{Z}/n\mathbb{Z}$ as follows:
            \begin{enumerate}
                  \item Explain why there are integers $u,v,a_1,b_1,n_1$ such that $au+nv=d$, $a=da_1$, $b=db_1$ and $n=dn_1$.\\
                        \textbf{Answer}: By Theorem 1.2, since $d=(a,n)$ there must exist $u,v\in\mathbb{Z}$ such that $au+nv=d$. By definition of gcd, $d=(a,n)$ divides both $a$ and $n$; in addition, it is given that $d$ divides $b$. Then by definition of divisibility there exists $a_1,b_1,n_1\in\mathbb{Z}$ such that $a=da_1$, $b=db_1$ and $n=dn_1$.
                  \item Show that each of $[ub_1],[ub_1+n_1],[ub_1+2n_1],\ldots,[ub_1+(d-1)n_1]$ is a solution of $[a]x=[b]$.\\
                        \textbf{Answer}: Define $k$ such that $0\leq k\leq d-1$, then we want to show that each of $[ub_1+kn_1]$ is a solution. By substitution we have $[a][ub_1+kn_1]=[aub_1+akn_1]=[(d-nv)b_1+da_1kn_1]=[db_1-nvb_1+dn_1a_1k=[b-n(vb_1+a_1k)]$. Note that $[b-n(vb_1+a_1k)]\equiv [b]$ (mod $n$), so each of $[ub_1+kn_1]$ is a solution of $[a]x=[b]$.
                  \item Show that the solutions listed in part (b) are all distinct.\\
                        \textbf{Answer}: Take distinct and arbitrary $k_1,k_2$ such that $0\leq k_1,k_2\leq d-1$, we want to show that $[ub_1+k_1n_1]\neq [ub_1+k_2n_1]$. By taking the difference we have $[ub_1+k_1n_1]-[ub_1+k_2n_1]=[(k_1-k_2)n_1]$. However $n=dn_1\nmid (k_1-k_2)n_1$ as we would need $(k_1-k_2)$ to be a multiple of $d$, which is not possible by our constraint $0\leq k_1,k_2\leq d-1$. Therefore the solutions listed in part (b) are distinct.
                  \item If $x=[r]$ is any solution of $[a]x=[b]$, show that $[r]=[ub_1]+[kn_1]$ for some integer $k$ with $0\leq k\leq d-1$.\\
                        \textbf{Answer}: Since $x=[r]$ is a solution as given, we have $[a][r]=[b]\implies [ar]-[b]=[0]$. As shown in part (b), $x=[ub_1]$ is also a solution, i.e. $[a][ub_1]=[b]$. By substitution we have $[ar]-[aub_1]=[0]$, therefore $[ar]\equiv [aub_1]$ (mod $n$) and $n|(ar-aub_1)$ by definition of congruence. Then since $n=dn_1$ and $a=da_1$, we have $dn_1|da_1(r-ub_1)\implies n_1|a_1(r-ub_1)$. Since $a_1$ and $n_1$ are constructed by factoring out the gcd of $a$ and $n$, we know that $(a_1,n_1)=1$. Therefore by Theorem 1.4 we have $n_1|(r-ub_1)$, i.e. there exist some $k\in\mathbb{Z}$ such that $kn_1=r-ub_1\implies r=ub_1+kn_1\implies [r]=[ub_1+kn_1]$.
            \end{enumerate}
      \item Use Problem 9 to solve the following equations:
            \begin{enumerate}
                  \item $15x=9$ in $\mathbb{Z}/18\mathbb{Z}$\\
                        \textbf{Answer}: We have $a=15,b=9$ and $n=18$, then $d=(a,n)=3$, $b_1=b/d=3$ and $n_1=n/d=6$. We can then take $u=-1$ and $v=1$ to have $au+nv=d$. Then by the problem 9(b), the solutions are $[-3],[-3+6],[-3+12]$ which are congruent to $[15],[3],[9]$ respectively.
                  \item $25x=10$ in $\mathbb{Z}/65\mathbb{Z}$.\\
                        \textbf{Answer}: We have $a=25,b=10$ and $n=65$, then $d=(a,n)=5$, $b_1=b/d=2$ and $n_1=n/d=13$. We can then take $u=-5$ and $v=2$ to have $au+nv=d$. Then by the problem 9(b), the solutions are $[-10],[-10+13],[-10+26],[-10+39],[-10+52]$ which are congruent to $[55],[3],[16],[29],[42]$ respectively.
            \end{enumerate}
\end{enumerate}

\end{document}