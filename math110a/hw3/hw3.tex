\documentclass{article}
\usepackage[margin=0.75in]{geometry}
\usepackage{enumitem}
\usepackage{setspace}
\usepackage{amsmath}
\usepackage{amssymb}
\usepackage{physics}
\usepackage{relsize}
\usepackage{multicol}

\title{Math 110A Homework 3}
\date{1/26/2021}
\author{Jiaping Zeng}

\begin{document}
\setstretch{1.35}
\maketitle

\begin{enumerate}
      \item \underline{A field of order 4:} Let $F=\{0,e,a,b\}$ with operations given by the following tables:
            \begin{center}
                  \begin{multicols}{2}
                        $\begin{array}{c|cccc}
                                    + & 0 & e & a & b \\
                                    \hline
                                    0 & 0 & e & a & b \\
                                    e & e & 0 & b & a \\
                                    a & a & b & 0 & e \\
                                    b & b & a & e & 0
                              \end{array}$

                        $\begin{array}{c|cccc}
                                    \cdot & 0 & e & a & b \\
                                    \hline
                                    0     & 0 & 0 & 0 & 0 \\
                                    e     & 0 & e & a & b \\
                                    a     & 0 & a & b & e \\
                                    b     & 0 & b & e & a
                              \end{array}$
                  \end{multicols}
            \end{center}
            Assume that associativity and distributivity hold for these operations. Show that $F$ is a field by checking the other axioms.\\
            \textbf{Answer}:
            \begin{enumerate}
                  \item (Closure for addition) as seen in the addition table, every entry is an element of $F$, therefore $F$ is closed under addition.
                  \item (Associative addition) we assume associativity holds.
                  \item (Commutative addition) the addition table is symmetric across the diagonal, therefore we have $a+b=b+a$ for any $a,b\in F$.
                  \item (Additive identity of zero element) the first row and the first column of the addition table is the same as their respective headers; in addition, the values are symmetric. Therefore the zero element satisfies $a+0=a=0+a$ for any $a\in F$.
                  \item Note that under every row header of the addition table, there is an entry with value 0. The same applies for every column header. Therefore $a+x=0$ always has a solution in $F$.
                  \item (Closure for multiplication) as seen in the multiplication table, every entry is an element of $F$, therefore $F$ is closed under multiplication.
                  \item (Associative multiplication) we assume associativity holds.
                  \item (Distributive laws) we assume distributivity holds.
            \end{enumerate}
      \item Which of the following sets of matrices are subrings of $M_2(\mathbb{R})$ (the ring of $2\cross 2$ matrices with real coefficients)? Which ones have an identity?
            \begin{enumerate}[start=2]
                  \item All matrices of the form $\begin{pmatrix}a&b\\0&c\end{pmatrix}$ with $a,b,c\in\mathbb{Z}$.\\
                        \textbf{Answer}: We have \[\begin{pmatrix}a&b\\0&c\end{pmatrix}-\begin{pmatrix}a'&b'\\0&c'\end{pmatrix}=\begin{pmatrix}a-a'&b-b'\\0&c-c'\end{pmatrix}\], so it is closed under substraction. We also have \[\begin{pmatrix}a&b\\0&c\end{pmatrix}\begin{pmatrix}a'&b'\\0&c'\end{pmatrix}=\begin{pmatrix}ad&ab'+bc'\\0&cc'\end{pmatrix}\], so it is also closed under multiplication. Therefore it is a subring by Theorem 3.6 with identity matrix $I_2$.
                  \item All matrices of the form $\begin{pmatrix}a&0\\a&0\end{pmatrix}$ with $a\in\mathbb{R}$.\\
                        \textbf{Answer}: We have \[\begin{pmatrix}a&0\\a&0\end{pmatrix}-\begin{pmatrix}a'&0\\a'&0\end{pmatrix}=\begin{pmatrix}a-a'&0\\a-a'&0\end{pmatrix}\], so it is closed under substraction. We also have \[\begin{pmatrix}a&0\\a&0\end{pmatrix}\begin{pmatrix}a'&0\\a'&0\end{pmatrix}=\begin{pmatrix}aa'&0\\aa'&0\end{pmatrix}\], so it is also closed under multiplication. Therefore it is a subring by Theorem 3.6 without an identity matrix since $I_2$ is not an element.
                  \item All matrices of the form $\begin{pmatrix}a&0\\0&0\end{pmatrix}$ with $a\in\mathbb{R}$.\\
                        \textbf{Answer}: We have \[\begin{pmatrix}a&0\\0&0\end{pmatrix}-\begin{pmatrix}a'&0\\0&0\end{pmatrix}=\begin{pmatrix}a-a'&0\\0&0\end{pmatrix}\], so it is closed under substraction. We also have \[\begin{pmatrix}a&0\\0&0\end{pmatrix}\begin{pmatrix}a'&0\\0&0\end{pmatrix}=\begin{pmatrix}aa'&0\\0&0\end{pmatrix}\], so it is also closed under multiplication. Therefore it is a subring by Theorem 3.6 without an identity matrix since $I_2$ is not an element.
            \end{enumerate}
      \item
            \begin{enumerate}
                  \item Define a new multiplication in $\mathbb{Z}$ by $ab=0$. Show that with ordinary addition and this new multiplication, $\mathbb{Z}$ is a commutative ring.\\
                        \textbf{Answer}: Since $\mathbb{Z}$ under ordinary addition and multiplication is a ring, we only need to check axioms 6-9 to see if it is a commutative ring under the new multiplication. Note that we have $ab=0\in\mathbb{Z}$ (axiom 6), $a(bc)=0=(ab)c$ (axiom 7), $a(b+c)=0=ab+ac$ and $(a+b)c=ac+bc$ (axiom 8) and $ab=0=ba$ (axiom 9) for any $a,b,c\in\mathbb{Z}$, so $\mathbb{Z}$ is a commutative ring under the new multiplication.
                  \item Define a new multiplication in $\mathbb{Z}$ by $ab=1$. With ordinary addition and this new multiplication, is $\mathbb{Z}$ a ring?\\
                        \textbf{Answer}: No because it fails axiom 8, e.g. $1(2+3)=1\neq 2=1\cdot 2+1\cdot 3$.
            \end{enumerate}
      \item Let $L$ be the set of positive real numbers. Define a new addition and multiplication on $L$ by $a\oplus b=ab$ and $a\otimes b=a^{\log b}$.
            \begin{enumerate}
                  \item Is $L$ a ring under these operations?\\
                        \textbf{Answer}:
                        \begin{enumerate}
                              \item (Closure for addition) since $L$ is closed under the standard multiplication, $L$ is closed under $\oplus$.
                              \item (Associative addition) since $L$ is associative under the standard multiplication, $L$ is associative under $\oplus$.
                              \item (Commutative addition) $a\oplus b=ab=ba=b\oplus a$, so $L$ is commutative under $\oplus$.
                              \item (Additive identity of zero element) since $1\oplus a=1\cdot a=a$ for any $a\in L$, $0_L=1$ is the additive identity.
                              \item We have $a\oplus\frac{1}{a}=1=0_L$ for any $a\in L$, so we can always take $x=\frac{1}{a}$ to be a solution of $a+x=0_L$.
                              \item (Closure for multiplication) since $a>0$, any power of $a$ is also positive. Therefore $a\otimes b\in L$ and $L$ is closed under $\otimes$.
                              \item (Associative multiplication) by $\log$ properties, we have $a\otimes (b\otimes c)=a^{(\log b^{\log c})}=(a^{\log b})^{\log c}=(a\otimes b)\otimes c$, so $L$ is associative under $\otimes$.
                              \item (Distributive laws) we have $a\otimes(b\oplus c)=a^{\log bc}=a^{\log b}\cdot a^{\log c}=a\otimes b\oplus a\otimes c$; similarly, we also have $(a\oplus b)\otimes c=(ab)^{\log c}=a^{\log b}\cdot b^{\log c}$, so distributivity holds.
                        \end{enumerate}
                  \item Is $L$ a commutative ring?\\
                        \textbf{Answer}: Take $a,b\in L$, we have $a\otimes b=a^{\log b}=e^{\log a^{\log b}}=e^{\log b\cdot \log a}=e^{\log b^{\log a}}=b^{\log a}$. Therefore $L$ is a commutative ring.
                  \item Is $L$ a field?\\
                        \textbf{Answer}: Note that we have $1_L=e$ since $a\otimes e=a^{\log e}=a=e^{\log a}=e\otimes a$, therefore $1_L=e\neq 1=0_L$. Now we want to show that $a\otimes x=1_L\implies a^{\log x}=e$ has a solution for $a\neq 0_L=1$. We can take $x=e^{(\log a)^{-1}}$ for $a\in L$, then $a\otimes x=e=1_L$. Therefore $L$ is a field by definition.
            \end{enumerate}
      \item Let $R$ be a ring, and let $Z(R)=\{a\in R|ar=ra\text{ for every }r\in R\}$. In other words, $Z(R)$ consists of all elements of $R$ that commute with every other element of $R$. Prove that $Z(R)$ is a subring of $R$. $Z(R)$ is called the \textbf{center} of $R$.\\
            \textbf{Answer}: Suppose we have $a,b\in Z(R)$ such that $ar=ra$ and $br=rb$ for any $r\in R$, then we also have $ar+br=ra+rb\implies (a+b)r=r(a+b)$. Therefore $(a+b)\in Z(R)$ and $Z(R)$ is closed under addition. Similarly, $ar\cdot br=abr^2=br\cdot ar$, so $ab\in Z(R)$ and $Z(R)$ is closed under multiplication. We have $0\in Z(R)$ since $0\cdot r=r\cdot 0$ for any $r$, and $a+x=0$ always has a solution in $Z(R)$ as $ar=ra\implies (-a)r=r(-a)\implies (-a)\in R(Z)$. Therefore $Z(R)$ is a subring of $R$ by Theorem 3.2.
      \item \underline{\bf The quaternions:} In the ring $M_2(\mathbb{C})$, let
            \begin{align*}
                  \textbf{1} & =\begin{pmatrix}1&0\\0&1\end{pmatrix}, &
                  \textbf{i} & =\begin{pmatrix}i&0\\0&-i\end{pmatrix}, &
                  \textbf{j} & =\begin{pmatrix}0&1\\-1&0\end{pmatrix}, &
                  \textbf{k} & =\begin{pmatrix}0&i\\i&0\end{pmatrix}.
            \end{align*}
            The product of a real number $r$ and a matrix is given by $r\begin{pmatrix}a&b\\c&d\end{pmatrix} = \begin{pmatrix}ra&rb\\rc&rd\end{pmatrix}$.

            The set $H$ of {\bf real quaternions} consists of all matrices of the form:
            \[
                  a\textbf{1}+b\textbf{i}+c\textbf{j}+d\textbf{k} = a\begin{pmatrix}1&0\\0&1\end{pmatrix}+b\begin{pmatrix}i&0\\0&-i\end{pmatrix}+c\begin{pmatrix}0&1\\-1&0\end{pmatrix}+d\begin{pmatrix}0&i\\i&0\end{pmatrix}
                  = \begin{pmatrix}a+bi&c+di\\-c+di&a-bi\end{pmatrix}
            \]
            where $a,b,c,d$ are real numbers.
            \begin{enumerate}[start=2]
                  \item Prove that $H$ is a ring.\\
                        \textbf{Answer}: Take arbitrary $a,b,c,d,a',b',c',d'\in\mathbb{R}$, then $a\textbf{1}+b\textbf{i}+c\textbf{j}+d\textbf{k}$ is an element of $H$. Similarly, so is $a'\textbf{1}+b'\textbf{i}+c'\textbf{j}+d'\textbf{k}$. Then the difference between the two is $(a-a')\textbf{1}+(b-b')\textbf{i}+(c-c')\textbf{j}+(d-d')\textbf{k}$ which satisfies the form so it is also in $H$, therefore $H$ is closed under subtraction.\\
                        For multiplication, we have \[\begin{pmatrix}a+bi&c+di\\-c+di&a-bi\end{pmatrix}\begin{pmatrix}a'+b'i&c'+d'i\\-c'+d'i&a'-b'i\end{pmatrix}\]\[=\begin{pmatrix}(aa'-bb'-cc'-dd')+(ab'+ba'+cd'-dc')i&(ac'-bd'+ca'+db')+(ad'+bc'-cb'+da')i\\-(ac'-bd'+ca'+db')+(ad'+bc'-cb'+da')i&(aa'-bb'-cc'-dd')-(ab'+ba'+cd'-dc')i\end{pmatrix}\] Note that the product also follows the form, so $H$ is also closed under multiplication.\\
                        Therefore $H$ is a ring by Theorem 3.6.
                  \item Show that $H$ is a division ring.\\
                        \textbf{Answer}: Note that the identity $I_2\in H$, so $H$ has an identity where $1_H\neq 0_H$. In addition, take the an arbitrary matrix in $H$, its determinant is $(a+bi)(a-bi)-(c+di)(-c+di)=a^2+b^2+c^2+d^2$, which is always positive unless $a=b=c=d=0$. Therefore every matrix in $H$ is invertible and is therefore a unit. Suppose we have an arbitrary matrix $M=a\textbf{1}+b\textbf{i}+c\textbf{j}+d\textbf{k}\in H$, take $x=ta-tb\textbf{i}-tc\textbf{j}-td\textbf{k}\in H$ where $t=1/(a^2+b^2+c^2+d^2)$. Then $Mx=ta^2\textbf{1}-tab\textbf{i}-tac\textbf{j}-tad\textbf{k}+tab\textbf{i}-tb^2\textbf{i}^2-tbc\textbf{ij}-tbd\textbf{ik}+tac\textbf{j}-tbc\textbf{ij}-tc^2\textbf{j}^2-tcd\textbf{jk}+tad\textbf{k}-tbd\textbf{ik}-tcd\textbf{jk}-td^2\textbf{k}^2$. By part (a), we have $Mx=ta^2\textbf{1}-tab\textbf{i}-tac\textbf{j}-tad\textbf{k}+tab\textbf{i}+tb^2\textbf{1}-tbc\textbf{k}+tbd\textbf{j}+tac\textbf{j}+tbc\textbf{k}+tc^2\textbf{1}-tcd\textbf{i}+tad\textbf{k}-tbd\textbf{j}+tcd\textbf{i}+td^2\textbf{1}=ta^2\textbf{1}+tb^2\textbf{1}+tc^2\textbf{1}+td^2\textbf{1}=t(a^2+b^2+c^2+d^2)\textbf{1}=\textbf{1}=1_H$. Therefore $ax=1_H$ always has a solution and $H$ is a division ring.
                  \item Show that the polynomial equation $x^2+1=0$ has \textit{infinitely many} solutions in $H$.\\
                        \textbf{Answer}: Take $M=b\textbf{i}+c\textbf{j}+d\textbf{k}$ with $b^2+c^2+d^2=1$, then by part (a), $M^2=b^2\textbf{i}^2+bc\textbf{ij}+bd\textbf{ik}+bc\textbf{ji}+c^2\textbf{j}^2+cd\textbf{jk}+bd\textbf{ki}+cd\textbf{kj}+d^2\textbf{k}^2=-b^2\textbf{1}+bc\textbf{k}-bd\textbf{j}-bc\textbf{k}-c^2\textbf{1}+cd\textbf{i}+bd\textbf{j}-cd\textbf{i}-d^2\textbf{1}=-b^2\textbf{1}-c^2\textbf{1}-d^2\textbf{1}=-(b^2+c^2+d^2)\textbf{1}=-\textbf{1}$. Therefore $M^2=-\textbf{1}\implies M^2+\textbf{1}=0$ for any  $M=b\textbf{i}+c\textbf{j}+d\textbf{k}$ satisfying $b^2+c^2+d^2=1$.
            \end{enumerate}
      \item Let $S$ and $T$ be subrings of a ring $R$. In (a) and (b), if the answer is "yes", prove it. If the answer is "no", give a counter-example.
            \begin{enumerate}
                  \item Is $S\cap T$ a subring of $R$?\\
                        \textbf{Answer}: Yes; take $a,b\in S\cap T$, since $a,b\in S$, $a-b\in S$ and $ab\in S$; since $a,b\in T$, $a-b\in T$ and $ab\in T$. Therefore $a-b\in S\cap T$ and $ab\in S\cap T$, so $S\cap T$ is a subring of $R$ by Theorem 3.6.
                  \item Is $S\cup T$ a subring of $R$?\\
                        \textbf{Answer}: No; take $S$ to be the ring of even integers and $T=\mathbb{Z}/5\mathbb{Z}$, note that $3\in S\cup T$ but does not have an inverse, so $S\cup T$ is not a ring.
            \end{enumerate}
      \item If $R$ is a \textit{finite} ring (i.e. a ring with only finitely many elements), prove that $R$ has characteristic $n$ for some $n>0$.\\
            \textbf{Answer}: Since $\mathbb{Z}$ is infinite and $R$ is finite, we must have some $a,b\in\mathbb{Z}$ such that $a\cdot 1_R=b\cdot 1_R$ where $a\neq b$, or else there would exist a unique element in $R$ for every element in $\mathbb{Z}$. Upon renaming we have $a>b$, then $a\cdot 1_R=b\cdot 1_R\implies a\cdot 1_R-b\cdot 1_R=0\implies (a-b)1_R=0$ where $(a-b)>0$. Now let $n=a-b$ and by definition of characteristic $R$ has characteristic $n$ for some $n>0$.
      \item Let $R$ be a ring with identity of characteristic $n>0$. \begin{enumerate}
                  \item Prove that $na=0_R$ for every $a\in R$.\\
                        \textbf{Answer}: Since $R$ has an identity, we have $a=1_R\cdot a$. Then $na=n\cdot 1_R\cdot a=0_R\cdot a=0_R$ by definition of characteristic.
                  \item If $R$ is an integral domain, prove that $n$ is prime.\\
                        \textbf{Answer}: By contradiction. Suppose $R$ is an integral domain and $n>0$ is not prime. Then we must have $n=1$ or $n=pq$ for some positive $pq\in\mathbb{Z}$ where $p,q$ are not $1$ or $n$. If $n=1$, we have $1\cdot 1_R=0_R$ which cannot be true by definition of integral domain. If $n=pq$, then we must have either $p\cdot 1_R=0_R$ or $q\cdot 1_R=0_R$. But since both $p$ and $q$ are positve integers smaller than $n$, $n$ is not the smallest positive integer that satisfies $n\cdot 1_R=0_R$ and therefore cannot be the characteristic. Therefore $n$ must be prime by contradiction.
            \end{enumerate}
\end{enumerate}

\end{document}