\documentclass{article}
\usepackage[margin=0.75in]{geometry}
\usepackage{enumitem}
\usepackage{setspace}
\usepackage{amsmath}
\usepackage{amssymb}
\usepackage{physics}
\usepackage{relsize}
\usepackage{multicol}

\title{Math 110A Homework 4}
\date{2/15/2021}
\author{Jiaping Zeng}

\begin{document}
\setstretch{1.35}
\maketitle

\begin{enumerate}
      \item For each pair of polynomials $f(x)$ and $g(x)$ below, use the Euclidean algorithm to compute the $\gcd(f(x),g(x))$, and to find polynomials $u(x)$ and $v(x)$ with $(f(x),g(x))=f(x)u(x)+g(x)v(x)$:
            \begin{enumerate}[start=2]
                  \item $f(x)=x^5+x^4+2x^3-x^2-x-2$ and $g(x)=x^4+2x^3+5x^2+4x+4$ in $\mathbb{Q}[x]$.\\
                        \textbf{Answer}: As follows:\\
                        $x^5+x^4+2x^3-x^2-x-2=(x^4+2x^3+5x^2+4x+4)(x-1)+(-x^3-x+2)$\\
                        $x^4+2x^3+5x^2+4x+4=(-x^3-x+2)(-x-2)+(4x^2+4x+8)$\\
                        $-x^3-x+2=(4x^2+4x+8)(-\frac{1}{4}x+\frac{1}{4})+0$\\
                        Therefore $\gcd(f(x),g(x))=x^2+x+2$; now let $u=\frac{1}{4}x+\frac{1}{2}$ and $v=-\frac{1}{4}x^2-\frac{1}{4}x+\frac{3}{4}$, we have $x^2+x+2=f(x)u(x)+g(x)v(x)$.
                  \item $f(x)=4x^4+2x^3+6x^2+4x+5$ and $g(x)=3x^3+5x^2+6x$ in $(\mathbb{Z}/7\mathbb{Z})[x]$.\\
                        \textbf{Answer}: As follows:\\
                        $4x^4+2x^3+6x^2+4x+5=(3x^3+5x^2+6x)(6x)+(5x^2+4x+5)$\\
                        $3x^3+5x^2+6x=(5x^2+4x+5)(2x+5)+(4x+3)$\\
                        $5x^2+4x+5=(4x+3)(3x+4)$
                        Since $2(4x+3)=x+6$ in $(\mathbb{Z}/7\mathbb{Z}[x])$, $\gcd(f(x),g(x))=x+6$; now let $u=3x+4$ and $v=3x^2+4x+2$, we have $x+6=f(x)u(x)+g(x)v(x)$.
                  \item $f(x)=x^3-ix^2+4x-4i$ and $g(x)=x^2+1$ in $\mathbb{C}[x]$.\\
                        \textbf{Answer}: As follows:\\
                        $x^3-ix^2+4x-4i=(x^2+1)(x-i)+(3x-3i)$\\
                        $3x-3i=3(x-i)$\\
                        Therefore $\gcd(f(x),g(x))=x-i$; now let $u=\frac{1}{3}$ and $v=-\frac{1}{3}(x-i)$, we have $x-i=f(x)u(x)+g(x)v(x)$.
            \end{enumerate}
      \item Express $x^4-4$ as a product of irreducibles in $\mathbb{Q}[x]$, $\mathbb{R}[x]$ and $\mathbb{C}[x]$.\\
            \textbf{Answer}:
            \begin{itemize}
                  \item [$\mathbb{Q}$:] $x^4-4=(x^2-2)(x^2+2)=(x^2-2)(x^2+2)$
                  \item [$\mathbb{R}$:] $x^4-4=(x^2-2)(x^2+2)=(x-\sqrt{2})(x+\sqrt{2})(x^2+2)$
                  \item [$\mathbb{C}$:] $x^4-4=(x^2-2)(x^2+2)=(x-\sqrt{2})(x+\sqrt{2})(x-\sqrt{2}i)(x+\sqrt{2}i)$
            \end{itemize}
      \item Use unique factorization to find the $\gcd$ in $\mathbb{C}[x]$ of $(x-3)^3(x-4)^4(x-i)^2$ and $(x-1)(x-3)(x-4)^3$.\\
            \textbf{Answer}: Since both polynomials are already expressed as products of irreducibles, the $\gcd$ is $(x-3)(x-4)^3$.
      \item
            \begin{enumerate}
                  \item Show that $x^2+2$ is irreducible in $(\mathbb{Z}/5\mathbb{Z})[x]$.\\
                        \textbf{Answer}: We can show that $x^2+2$ has no roots in $\mathbb{Z}/5\mathbb{Z}$ (i.e. $x^2+2=0$ has no solutions):
                        \begin{center}
                              \begin{tabular}{c|c|c}
                                    $x$   & $x^2+2$                  & is solution? \\
                                    \hline
                                    $[0]$ & $[0]^2+[2]=[0]+[2]=[2]$  & no           \\
                                    $[1]$ & $[1]^2+[2]=[1]+[2]=[3]$  & no           \\
                                    $[2]$ & $[2]^2+[2]=[4]+[2]=[1]$  & no           \\
                                    $[3]$ & $[3]^2+[2]=[9]+[2]=[1]$  & no           \\
                                    $[4]$ & $[4]^2+[2]=[16]+[2]=[3]$ & no
                              \end{tabular}
                        \end{center}
                        Since $x^2+2$ has no roots in $\mathbb{Z}/5\mathbb{Z}$, it is irreducible.
                  \item Factor $x^4-4$ as a product of irreducibles in $(\mathbb{Z}/5\mathbb{Z})[x]$.\\
                        \textbf{Answer}: We have $x^4-4=(x^2+2)(x^2-2)$; we can verify that it cannot be reduced further by checking if $(x^2+2)$ and $(x^2-2)$ have roots in $(\mathbb{Z}/5\mathbb{Z})[x]$:
                        \begin{center}
                              \begin{tabular}{c|c|c}
                                    $x$   & $x^2+2$                  & is solution? \\
                                    \hline
                                    $[0]$ & $[0]^2+[2]=[0]+[2]=[2]$  & no           \\
                                    $[1]$ & $[1]^2+[2]=[1]+[2]=[3]$  & no           \\
                                    $[2]$ & $[2]^2+[2]=[4]+[2]=[1]$  & no           \\
                                    $[3]$ & $[3]^2+[2]=[9]+[2]=[1]$  & no           \\
                                    $[4]$ & $[4]^2+[2]=[16]+[2]=[3]$ & no
                              \end{tabular}
                        \end{center}
                        \begin{center}
                              \begin{tabular}{c|c|c}
                                    $x$   & $x^2-2$                  & is solution? \\
                                    \hline
                                    $[0]$ & $[0]^2-[2]=[0]-[2]=[3]$  & no           \\
                                    $[1]$ & $[1]^2-[2]=[1]-[2]=[4]$  & no           \\
                                    $[2]$ & $[2]^2-[2]=[4]-[2]=[2]$  & no           \\
                                    $[3]$ & $[3]^2-[2]=[9]-[2]=[2]$  & no           \\
                                    $[4]$ & $[4]^2-[2]=[16]-[2]=[4]$ & no
                              \end{tabular}
                        \end{center}
                        Since neither has roots in $(\mathbb{Z}/5\mathbb{Z})[x]$, $x^4-4=(x^2+2)(x^2-2)$ cannot be reduced further.
            \end{enumerate}
      \item Use the factor theorem to show that $x^7-x$ factors in $(\mathbb{Z}/7\mathbb{Z})[x]$ as $x(x-1)(x-2)(x-3)(x-4)(x-5)(x-6)$, \textit{without} doing any polynomial multiplication.\\
            \textbf{Answer}: We can do so by showing that $x=0,1,2,3,4,5,6$ are all solutions to $x^7-x=0$ in $\mathbb{Z}/\mathbb{Z}$ as follows:
            \begin{center}
                  \begin{tabular}{c|c|c}
                        $x$   & $x^7-x$                      & is solution? \\
                        \hline
                        $[0]$ & $[0]^7-[0]=[0]-[0]=[0]$      & yes          \\
                        $[1]$ & $[1]^7-[1]=[1]-[1]=[0]$      & yes          \\
                        $[2]$ & $[2]^7-[2]=[128]-[2]=[0]$    & yes          \\
                        $[3]$ & $[3]^7-[3]=[2187]-[3]=[0]$   & yes          \\
                        $[4]$ & $[4]^7-[4]=[16384]-[4]=[0]$  & yes          \\
                        $[5]$ & $[5]^7-[5]=[78125]-[5]=[0]$  & yes          \\
                        $[6]$ & $[6]^7-[6]=[279936]-[6]=[0]$ & yes
                  \end{tabular}
            \end{center}
            Therefore $x=0,1,2,3,4,5,6$ are all roots, so $x,(x-1),(x-2),(x-3),(x-4),(x-5),(x-6)$ are all factors by the factor theorem.
      \item Determine if the given polynomial is irreducible:
            \begin{enumerate}
                  \item $x^2-7$ in $\mathbb{Q}[x]$.\\
                        \textbf{Answer}: $x^2-7=0\implies x^2=7$ which has no solution in $\mathbb{Q}$, therefore $x^2-7$ is irreducible in $\mathbb{Q}[x]$.
                  \item $2x^3+x^2+2x+2$ in $(\mathbb{Z}/5\mathbb{Z})[x]$.\\
                        \textbf{Answer}:
                        \begin{center}
                              \begin{tabular}{c|c|c}
                                    $x$   & $2x^3+x^2+2x+2$                                & is solution? \\
                                    \hline
                                    $[0]$ & $2[0]^3+[0]^2+2[0]+[2]=[0]+[0]+[0]+[2]=[2]$    & no           \\
                                    $[1]$ & $2[1]^3+[1]^2+2[1]+[2]=[2]+[1]+[2]+[2]=[2]$    & no           \\
                                    $[2]$ & $2[2]^3+[2]^2+2[2]+[2]=[16]+[4]+[4]+[2]=[1]$   & no           \\
                                    $[3]$ & $2[3]^3+[3]^2+2[3]+[2]=[54]+[9]+[6]+[2]=[1]$   & no           \\
                                    $[4]$ & $2[4]^3+[4]^2+2[4]+[2]=[128]+[16]+[8]+[2]=[4]$ & no
                              \end{tabular}
                        \end{center}
                        Therefore $2x^3+x^2+2x+2$ is irreducible $(\mathbb{Z}/5\mathbb{Z})[x]$.
                  \item $x^4+x^2+1$ in $(\mathbb{Z}/3\mathbb{Z})[x]$.\\
                        \textbf{Answer}:
                        \begin{center}
                              \begin{tabular}{c|c|c}
                                    $x$   & $x^4+x^2+1$                        & is solution? \\
                                    \hline
                                    $[0]$ & $[0]^4+[0]^2+[1]=[0]+[0]+[1]=[1]$  & no           \\
                                    $[1]$ & $[1]^4+[1]^2+[1]=[1]+[1]+[1]=[0]$  & yes           \\
                                    $[2]$ & $[2]^4+[2]^2+[1]=[16]+[4]+[1]=[0]$ & yes
                              \end{tabular}
                        \end{center}
                        Therefore $x^4+x^2+1$ is not irreducible since $x=1,2$ are roots.
            \end{enumerate}
      \item Let $\varphi:\mathbb{C}\rightarrow\mathbb{C}$ be an isomorphism of rings such that $\varphi(a)=a$ for all $a\in\mathbb{Q}$. Suppose that $r\in\mathbb{C}$ is a root of $f(x)\in\mathbb{Q}[x]$. Prove that $\varphi(r)$ is also a root of $f(x)$.\\
            \textbf{Answer}: Since $f(x)\in\mathbb{Q}[x]$ and $\varphi$ is a ring isomorphism, we have $f(r)=a_0+a_1r+\ldots+a_nr^n\implies 0=\varphi(f(r))=a_0+a_1\varphi(r)+\ldots+a_n\varphi(a_n)=f(\varphi(r))$, therefore $\varphi(r)$ is a root by definition.
      \item
            \begin{enumerate}
                  \item Prove that the following version of the division algoirthm holds in $\mathbb{Z}[i]$: \textit{For any $\alpha,\beta\in\mathbb{Z}[i]$ with $\beta\neq 0$, there is some not necessarily unique $q,r\in\mathbb{Z}[i]$ with $\alpha=\beta q+r$ and $N(r)<N(\beta)$}.\\
                        \textbf{Answer}: Let $z=\frac{\alpha}{\beta}\in\mathbb{C}$ such that $\frac{r}{\beta}=z-q$, we want to show that there exists some $q\in\mathbb{Z}[i]$ such that $\abs{z-q}<1$. We can do so by taking the four points in $\mathbb{Z}[i]$ closest to $z$ as follows: let $z=x+yi$, and let $x_1,x_2,y_1,y_2\in\mathbb{Z}$ such that $x_1\leq x<x_2,y_1\leq y<y_2$ and $x_2-x_1=y_2-y_1=1$ (visually, the four points $(x_1,y_1),(x_1,y_2),(x_2,y_1),(x_2,y_2)$ form a square that contains $z$). Now if we divide the square into four smaller squares evenly, $z$ must be in or on one of the smaller squares. Since each of the smaller squares contains a point from $\mathbb{Z}[i]$, the distance between $z$ to a point from $\mathbb{Z}[i]$ must be smaller than the diagonal of the smaller square, which is $\frac{\sqrt{2}}{2}<1$. Therefore $\abs{z-q}<1$.
                  \item Give an example to show that the $q$ and $r$ you found in part (a) do not need to be unique.\\
                        \textbf{Answer}: Let $\alpha=1+2i$ and $\beta=2+4i$, then $\alpha=1\cdot\beta+(-1-2i)$ with $q=1$ and $r=-1-2i$. Similarly, $\alpha=0\cdot\beta+(1+2i)$ with $q=0$ and $r=1+2i$. Therefore $q$ and $r$ do not need to be unique.
                  \item If $\alpha,\beta\in\mathbb{Z}[i]$ are irreducibles and $\alpha|\beta$, prove that $\alpha$ and $\beta$ are associates.\\
                        \textbf{Answer}: Since $\alpha|\beta$, there is some $\gamma\in\mathbb{Z}[i]$ such that $\alpha=\beta\gamma$. Then since $\alpha$ is irreducible, either $\beta$ or $\gamma$ is a unit. Since $\beta$ is irreducible and therefore cannot be a unit by definition, $\gamma$ is a unit. Then $\alpha$ and $\beta$ are associates by definition.
            \end{enumerate}
      \item Let $p>0$ be a prime number in $\mathbb{Z}$. Notice that it is possible for a prime in $\mathbb{Z}$ to be no longer be irreducible in $\mathbb{Z}[i]$. For example, $2=(1+i)(1-i)$ or $13=(2+3i)(2-3i)$. However, some primes are still irreducible in $\mathbb{Z}[i]$.
            \begin{enumerate}
                  \item Prove that if $p=a^2+b^2$ for $a,b\in\mathbb{Z}$, then $p$ is reducible in $\mathbb{Z}[i]$.\\
                        \textbf{Answer}: We have $p=a^2+b^2=(a+bi)(a-bi)$ for $a,b\in\mathbb{Z}$. Note that $a,b\neq 0$ since $p$ is prime (or else we would have $p=a^2\implies a|p$ or $p=b^2\implies b|p$); therefore neither $a+bi$ nor $a-bi$ is a unit so $p$ is reducible into $(a+bi)(a-bi)$.
                  \item Prove that if $p$ is prime in $\mathbb{Z}$ but is reducible in $\mathbb{Z}[i]$ then $p=a^2+b^2$ for some $a,b\in\mathbb{Z}$.\\
                        \textbf{Answer}: Since $p$ is reducible, we have $p=\alpha\beta$ for some $\alpha,\beta\in\mathbb{Z}[i]$ that are not units. Then $N(\alpha)N(\beta)=N(p)=p^2\implies N(\alpha)=N(\beta)=p$ since $N(\alpha)\neq 1$ and $N(\beta)\neq 1$. Now let $\alpha=a+bi$ where $a,b\in\mathbb{Z}$, then $p=N(\alpha)=N(a+bi)=a^2+b^2\implies p=a^2+b^2$.
            \end{enumerate}
\end{enumerate}
\end{document}