\documentclass{article}
\usepackage[margin=1in]{geometry}
\usepackage{enumitem}
\usepackage{setspace}
\usepackage{amsmath}
\usepackage{amssymb}
\usepackage{physics}
\usepackage{relsize}
\usepackage{multicol}
\usepackage{chngpage}

\title{Math 110A Homework 7}
\date{3/1/2021}
\author{Jiaping Zeng}

\begin{document}
\setstretch{1.35}
\maketitle

\begin{enumerate}
      \item Write out the addition and multiplication tables for the congruence class ring $F[x]/(p(x))$. In each case, is $F[x]/(p(x))$ a field?
            \begin{enumerate}
                  \item $F=\mathbb{Z}/2\mathbb{Z},p(x)=x^3+x+1$\\
                        \textbf{Answer}:
                        \begin{adjustwidth}{-1in}{-1in}
                              \begin{center}
                                    \begin{tabular}{c|c c c c c c c c}
                                          $+$         & $[0]$       & $[1]$       & $[x]$       & $[x+1]$     & $[x^2]$     & $[x^2+1]$   & $[x^2+x]$   & $[x^2+x+1]$ \\
                                          \hline
                                          $[0]$       & $[0]$       & $[1]$       & $[x]$       & $[x+1]$     & $[x^2]$     & $[x^2+1]$   & $[x^2+x]$   & $[x^2+x+1]$ \\
                                          $[1]$       & $[1]$       & $[0]$       & $[x+1]$     & $[x]$       & $[x^2+1]$   & $[x^2]$     & $[x^2+x+1]$ & $[x^2+x]$   \\
                                          $[x]$       & $[x]$       & $[x+1]$     & $[0]$       & $[1]$       & $[x^2+x]$   & $[x^2+x+1]$ & $[x^2+$     & $[x^2+1]$   \\
                                          $[x+1]$     & $[x+1]$     & $[x]$       & $[1]$       & $[0]$       & $[x^2+x+1]$ & $[x^2+x]$   & $[x^2+1]$   & $[x^2]$     \\
                                          $[x^2]$     & $[x^2]$     & $[x^2+1]$   & $[x^2+x]$   & $[x^2+x+1]$ & $[0]$       & $[1]$       & $[x]$       & $[x+1]$     \\
                                          $[x^2+1]$   & $[x^2+1]$   & $[x^2]$     & $[x^2+x+1]$ & $[x^2+x]$   & $[1]$       & $[0]$       & $[x+1]$     & $[x]$       \\
                                          $[x^2+x]$   & $[x^2+x]$   & $[x^2+x+1]$ & $[x^2]$     & $[x^2+1]$   & $[x]$       & $[x+1]$     & $[0]$       & $[1]$       \\
                                          $[x^2+x+1]$ & $[x^2+x+1]$ & $[x^2+x]$   & $[x^2+1]$   & $[x^2]$     & $[x+1]$     & $[x]$       & $[1]$       & $[0]$
                                    \end{tabular}\\
                                    \begin{tabular}{c|c c c c c c c c}
                                          $\cdot$     & $[0]$ & $[1]$       & $[x]$       & $[x+1]$     & $[x^2]$     & $[x^2+1]$   & $[x^2+x]$   & $[x^2+x+1]$ \\
                                          \hline
                                          $[0]$       & $[0]$ & $[0]$       & $[0]$       & $[0]$       & $[0]$       & $[0]$       & $[0]$       & $[0]$       \\
                                          $[1]$       & $[0]$ & $[1]$       & $[x]$       & $[x+1]$     & $[x^2]$     & $[x^2+1]$   & $[x^2+x]$   & $[x^2+x+1]$ \\
                                          $[x]$       & $[0]$ & $[x]$       & $[x^2]$     & $[x^2+x]$   & $[x+1]$     & $[1]$       & $[x^2+x+1]$ & $[x^2+1]$   \\
                                          $[x+1]$     & $[0]$ & $[x+1]$     & $[x^2+x]$   & $[x^2+1]$   & $[x^2+x+1]$ & $[x^2+$     & $[1]$       & $[x]$       \\
                                          $[x^2]$     & $[0]$ & $[x^2]$     & $[x+1]$     & $[x^2+x+1]$ & $[x^2+x]$   & $[x]$       & $[x^2+x]$   & $[1]$       \\
                                          $[x^2+1]$   & $[0]$ & $[x^2+1]$   & $[1]$       & $[x^2]$     & $[x]$       & $[x^2+x+1]$ & $[x+1]$     & $[x^2+x]$   \\
                                          $[x^2+x]$   & $[0]$ & $[x^2+x]$   & $[x^2+x+1]$ & $[1]$       & $[x^2+1]$   & $[x+1]$     & $[x]$       & $[x^2]$     \\
                                          $[x^2+x+1]$ & $[0]$ & $[x^2+x+1]$ & $[x^2+1]$   & $[x]$       & $[1]$       & $[x^2+x]$   & $[x^2]$     & $[x+16]$
                                    \end{tabular}
                              \end{center}
                        \end{adjustwidth}
                  \item $F=\mathbb{Z}/3\mathbb{Z},p(x)=x^2+1$\\
                        \textbf{Answer}:
                        \begin{adjustwidth}{-1in}{-1in}
                              \begin{center}
                                    \begin{tabular}{c|c c c c c c c c c}
                                          $+$      & $[0]$    & $[1]$    & $[2]$    & $[x]$    & $[x+1]$  & $[x+2]$  & $[2x]$   & $[2x+1]$ & $[2x+2]$ \\
                                          \hline
                                          $[0]$    & $[0]$    & $[1]$    & $[2]$    & $[x]$    & $[x+1]$  & $[x+2]$  & $[2x]$   & $[2x+1]$ & $[2x+2]$ \\
                                          $[1]$    & $[1]$    & $[2]$    & $[0]$    & $[x+1]$  & $[x+2]$  & $[x]$    & $[2x+1]$ & $[2x+2]$ & $[2x]$   \\
                                          $[2]$    & $[2]$    & $[0]$    & $[1]$    & $[x+2]$  & $[x]$    & $[x+1]$  & $[2x+2]$ & $[2x]$   & $[2x+1]$ \\
                                          $[x]$    & $[x]$    & $[x+1]$  & $[x+2]$  & $[2x]$   & $[2x+1]$ & $[2x+2]$ & $[0]$    & $[1]$    & $[2]$    \\
                                          $[x+1]$  & $[x+1]$  & $[x+2]$  & $[x]$    & $[2x+1]$ & $[2x+2]$ & $[2x]$   & $[1]$    & $[2]$    & $[0]$    \\
                                          $[x+2]$  & $[x+2]$  & $[x]$    & $[x+1]$  & $[2x+2]$ & $[2x]$   & $[2x+1]$ & $[2]$    & $[0]$    & $[1]$    \\
                                          $[2x]$   & $[2x]$   & $[2x+1]$ & $[2x+2]$ & $[0]$    & $[1]$    & $[2]$    & $[x]$    & $[x+1]$  & $[x+2]$  \\
                                          $[2x+1]$ & $[2x+1]$ & $[2x+2]$ & $[2x]$   & $[1]$    & $[2]$    & $[0]$    & $[x+1]$  & $[x+2]$  & $[x]$    \\
                                          $[2x+2]$ & $[2x+2]$ & $[2x]$   & $[2x+1]$ & $[2]$    & $[0]$    & $[1]$    & $[x+2]$  & $[x]$    & $[x+1]$
                                    \end{tabular}\\
                                    \begin{tabular}{c|c c c c c c c c c}
                                          $\cdot$  & $[0]$ & $[1]$    & $[2]$    & $[x]$    & $[x+1]$  & $[x+2]$  & $[2x]$   & $[2x+1]$ & $[2x+2]$ \\
                                          \hline
                                          $[0]$    & $[0]$ & $[0]$    & $[0]$    & $[0]$    & $[0]$    & $[0]$    & $[0]$    & $[0]$    & $[0]$    \\
                                          $[1]$    & $[0]$ & $[1]$    & $[2]$    & $[x]$    & $[x+1]$  & $[x+2]$  & $[2x]$   & $[2x+1]$ & $[2x+2]$ \\
                                          $[2]$    & $[0]$ & $[2]$    & $[1]$    & $[2x]$   & $[2x+2]$ & $[2x+1]$ & $[x]$    & $[x+2]$  & $[x+1]$  \\
                                          $[x]$    & $[0]$ & $[x]$    & $[2x]$   & $[2]$    & $[x+2]$  & $[2x+2]$ & $[1]$    & $[x+1]$  & $[2x+1]$ \\
                                          $[x+1]$  & $[0]$ & $[x+1]$  & $[2x+2]$ & $[x+2]$  & $[2x]$   & $[1]$    & $[2x+1]$ & $[2]$    & $[x]$    \\
                                          $[x+2]$  & $[0]$ & $[x+2]$  & $[2x+1]$ & $[2x+2]$ & $[1]$    & $[x]$    & $[x+1]$  & $[2x]$   & $[2]$    \\
                                          $[2x]$   & $[0]$ & $[2x]$   & $[x]$    & $[1]$    & $[2x+1]$ & $[x+1]$  & $[2]$    & $[2x+2]$ & $[x+2]$  \\
                                          $[2x+1]$ & $[0]$ & $[2x+1]$ & $[x+2]$  & $[x+1]$  & $[2]$    & $[2x]$   & $[2x+2]$ & $[x]$    & $[1]$    \\
                                          $[2x+2]$ & $[0]$ & $[2x+2]$ & $[x+1]$  & $[2x+1]$ & $[x]$    & $[2]$    & $[x+2]$  & $[1]$    & $[2x]$
                                    \end{tabular}
                              \end{center}
                        \end{adjustwidth}
                  \item $F=\mathbb{Z}/2\mathbb{Z},p(x)=x^2+1$\\
                        \textbf{Answer}:
                        \begin{center}
                              \begin{tabular}{c|c c c c}
                                    $+$     & $[0]$   & $[1]$   & $[x]$   & $[x+1]$ \\
                                    \hline
                                    $[0]$   & $[0]$   & $[1]$   & $[x]$   & $[x+1]$ \\
                                    $[1]$   & $[1]$   & $[0]$   & $[x+1]$ & $[x]$   \\
                                    $[x]$   & $[x]$   & $[x+1]$ & $[0]$   & $[1]$   \\
                                    $[x+1]$ & $[x+1]$ & $[x]$   & $[1]$   & $[0]$
                              \end{tabular}\\
                              \begin{tabular}{c|c c c c}
                                    $\cdot$ & $[0]$ & $[1]$   & $[x]$   & $[x+1]$ \\
                                    \hline
                                    $[0]$   & $[0]$ & $[0]$   & $[0]$   & $[0]$   \\
                                    $[1]$   & $[0]$ & $[1]$   & $[x]$   & $[x+1]$ \\
                                    $[x]$   & $[0]$ & $[x]$   & $[1]$   & $[x+1]$ \\
                                    $[x+1]$ & $[0]$ & $[x+1]$ & $[x+1]$ & $[0]$
                              \end{tabular}
                        \end{center}
            \end{enumerate}
      \item Find a fourth-degree polynomial in $(\mathbb{Z}/2\mathbb{Z})[x]$ whose roots are the four elements of the field $(\mathbb{Z}/2\mathbb{Z})[x]/(x^2+x+1)$.\\
            \textbf{Answer}: As shown in Example 3, the four elements of $(\mathbb{Z}/2\mathbb{Z})[x]/(x^2+x+1)$ are $[0],[1],[x],[x+1]$. Then we have $x(x+1)(x^2+x+1)=x^4+2x^3+2x^2+x\equiv x^4+x$. Therefore the roots of $p(x)=x^4+x$ are the four elements of $(\mathbb{Z}/2\mathbb{Z})[x]/(x^2+x+1)$.
      \item
            \begin{enumerate}
                  \item Show that $(\mathbb{Z}/2\mathbb{Z})[x]/(x^3+x+1)$ is a field.\\
                        \textbf{Answer}: By substitution, neither of $0$ or $1$ is a root of $x^3+x+1$ ($p(0)=1,p(1)=1$). Therefore $x^3+x+1$ is irreducible in $\mathbb{Z}/2\mathbb{Z}$ by Corollary 4.19. Then by Theorem 5.10 $(\mathbb{Z}/2\mathbb{Z})[x]/(x^3+x+1)$ is a field.
                  \item Show that $(\mathbb{Z}/2\mathbb{Z})[x]/(x^3+x+1)$ contains all three roots of $x^3+x+1$.\\
                        \textbf{Answer}: $[x],[x^2],[x^2+x]$ are roots of $x^3+x+1$ in $(\mathbb{Z}/2\mathbb{Z})[x]/(x^3+x+1)$. Therefore $(\mathbb{Z}/2\mathbb{Z})[x]/(x^3+x+1)$ contains all three roots of $x^3+x+1$.
            \end{enumerate}
      \item Show that $\mathbb{Q}[x]/(x^2-2)$ is not isomorphic to $\mathbb{Q}[x]/(x^2-3)$.\\
            \textbf{Answer}: Suppose there is a solution to $a^2=2$ in $\mathbb{Q}[x]/(x^2-3)$, which would imply that $\sqrt{2}\in\mathbb{Q}$ which is a contradiction. Therefore $\mathbb{Q}[x]/(x^2-2)$ is not isomorphic to $\mathbb{Q}[x]/(x^2-3)$.
      \item Show that $\mathbb{Q}[x]/(x^2-2)$ is isomorphic to $\mathbb{Q}[x]/(x^2+2x-1)$.\\
            \textbf{Answer}: Let $f(x)=x+1$ and $\varphi(f(x))=f(x+1)$, then $\varphi^{-1}(f(x))=f(x-1)$. Note that $\varphi(x^2-2)=(x+1)^2-2=x^2+2x-1$ and $\varphi(x^2+2x-1)=(x-1)^2+2(x-1)-1=x^2-2$. Therefore $\mathbb{Q}[x]/(x^2-2)$ is isomorphic to $\mathbb{Q}[x]/(x^2+2x-1)$.
      \item
            \begin{enumerate}
                  \item Show that the set $I=\{(k,0)|k\in\mathbb{Z}\}$ is an ideal in the ring $\mathbb{Z}\cross\mathbb{Z}$.\\
                        \textbf{Answer}: Take $(p,q)\in\mathbb{Z}\cross\mathbb{Z}$ and $(k,0)\in I$, we have $(p,q)(k,0)=(kp,0)\in I$ and $(k,0)(p,q)=(kp,0)\in I$. Therefore $I$ is an ideal.
                  \item Show that the set $I=\{(k,k)|k\in\mathbb{Z}\}$ is \textit{not} an ideal in the ring $\mathbb{Z}\cross\mathbb{Z}$.\\
                        \textbf{Answer}: Take $(1,2)\in\mathbb{Z}\cross\mathbb{Z}$ and $(k,k)\in I$, we have $(1,2)(k,k)=(k,2k)$ which is not in $I$ for nonzero $k$. Therefore $I$ is not an ideal.
            \end{enumerate}
      \item List all distinct principal ideals in each ring:
            \begin{enumerate}
                  \item $\mathbb{Z}/5\mathbb{Z}$\\
                        \textbf{Answer}: $(0)=\{0\}$.
                  \item $\mathbb{Z}/9\mathbb{Z}$\\
                        \textbf{Answer}: $(0)=\{0\},(3)=\{3\}$.
                  \item $\mathbb{Z}/12\mathbb{Z}$\\
                        \textbf{Answer}: $(0)=\{0\},(2)=\{2,4,6,8,10,0\},(3)=\{3,6,9,0\},(4)=\{4,8,0\},(6)=\{6,0\}$.
            \end{enumerate}
      \item
            \begin{enumerate}
                  \item If $I$ and $J$ are ideals of $R$, prove that $I\cap J$ is also an ideal.\\
                        \textbf{Answer}: Take $a,b\in I\cap J$ and $r\in R$, then $a-b\in I$ and $a-b\in J$ since $I$ and $J$ are ideals, so $a-b\in I\cap J$. We also have $ar\in I$ and $ra\in I$ since $I$ is an ideal; Similarly, we also have $ar\in J$ and $ra\in J$ since $J$ is an ideal. Therefore $ar\in I\cap J$ and $ra\in I\cap J$, so $I\cap J$ is an ideal by Theorem 6.1.
                  \item If $\{I_k\}_{k\in S}$ is a (possibly infinite) family of ideals in $R$, prove that the intersection $\bigcap_{k\in S}I_k$ is also an ideal in $R$.\\
                        \textbf{Answer}: By induction on the number of elements $n$;\\
                        Base case: $n=2$, $\{I_1,I_2\}$ is a family of ideals in $R$, then $I_1\cap I_2$ is an ideal by part (a).\\
                        Inductive step: Suppose that the statement holds for $n-1$ elements, we want to show that it will also hold for $n$ elements. Let $A=\bigcap_{k=1}^{n-1}I_k$, then $A$ is an ideal by inductive hypothesis. Then $A\cap I_n=\bigcap_{k=1}^nI_k$ is also an ideal by part (a).\\
                        Therefore $\bigcap_{k\in S}I_k$ is an ideal in $R$.
                  \item Give an example in $\mathbb{Z}$ to prove that if $I$ and $J$ are ideals, that $I\cup J$ might not be an ideal (or even a subring).\\
                        \textbf{Answer}: Take $I=2\mathbb{Z}$ and $J=3\mathbb{Z}$, then $2\in I$ and $3\in J$ but $3-2=1\notin I\cup J$. So $I\cup J$ is not a subring.
            \end{enumerate}
\end{enumerate}
\end{document}