\documentclass{article}
\usepackage[margin=1in]{geometry}
\usepackage{enumitem}
\usepackage{setspace}
\usepackage{amsmath}
\usepackage{amssymb}
\usepackage{physics}
\usepackage{relsize}
\usepackage{multicol}
\usepackage{chngpage}

\title{Math 110A Homework 8}
\date{3/7/2021}
\author{Jiaping Zeng}

\begin{document}
\setstretch{1.35}
\maketitle

\begin{enumerate}
      \item Let $R$ be an ring with identity and let $I$ be an ideal of $R$.
            \begin{enumerate}
                  \item If $1_R\in I$, prove that $I=R$.\\
                        \textbf{Answer}: Take any $r\in R$, we must have $1_R\cdot r=r\in I$ by definition of ideal. Therefore every element of $R$ is in $I$, so $I=R$.
                  \item If $I$ contains a unit, prove that $I=R$.\\
                        \textbf{Answer}: Let $a\in I$ be a unit, then by definition $ax=1_R$ has a solution in $R$. Then by definition of ideal we have $ax=1_R\in I$, therefore $I=R$ by part (a).
                  \item If $I$ is an ideal in a field $F$, prove that either $I=(0_F)$ or $I=F$.\\
                        \textbf{Answer}: By definition of field, $1_F\neq 0_F$. Then, if $1_F\in I$, we have $I=F$ by part (a); if not, we can only have $I=(0_F)$ or else we would again have $I=F$ by part (b) since every nonzero element is a unit.
            \end{enumerate}
      \item Let $I$ and $J$ be ideals in $R$.
            \begin{enumerate}
                  \item Prove that the set $K=\{a+b\mid a\in I, b\in J\}$ is an ideal in $R$ that contains both $I$ and $J$. $K$ is called the \textbf{sum} of $I$ and $J$, and is denoted $I+J$.\\
                        \textbf{Answer}: Take $a,b\in I$ and $c,d\in J$, then $a+c\in K$ and $b+d\in K$. We have $(a+c)-(b+d)=(a-b)+(c-d)\in K$ since $a-b\in I$ and $c-d\in J$ by Theorem 6.1. We also have $r(a+c)\in K$ and $(a+c)\in K$ since $r(a+c)=ra+rc$ and $(a+c)r=ar+cr$, where $ra,ar\in I$ and $rc,cr\in J$ by Theorem 6.1. Then $K$ satisfies both conditions of Theorem 6.1 and is therefore an ideal. It also contains both $I$ and $J$ upon taking $b=0$ or $a=0$ respectively in the definition.
                  \item Is the set $K=\{ab\mid a\in I, b\in J\}$ always an ideal in $R$?\\
                        \textbf{Answer}: No; take $R=\mathbb{Z}$, $I=2\mathbb{Z}$ and $J=3\mathbb{Z}$. We have $4\in I\subset K$ and $9\in J\subset K$, so by Theorem 6.1 we must have $9-4=5\in IJ$ which is not true.
                  \item Let $IJ$ denote the set of all possible finite sums of elements of the form $ab$ (with $a\in I, b\in J$), that is: \[IJ=\{a_1b_1+a_2b_2+\cdots+a_nb_n\mid n\geq 1, a_k\in I, b_k\in J\}.\] Prove that $IJ$ is an ideal of $R$. $IJ$ is called the \textbf{product} of $I$ and $J$.\\
                        \textbf{Answer}: Take $p,q\in IJ$ with $p=a_1b_1+a_2b_2+\cdots+a_nb_n$ and $q=c_1d_1+c_2d_2+\cdots+c_nd_n$, we have $p-q=a_1b_1+a_2b_2+\cdots+a_nb_n-c_1d_1-c_2d_2-\cdots-c_nd_n$ which is in $IJ$ since each $a_kb_k$ and $-c_kd_k$ is in $IJ$. Now take $r\in R$, we have $rp=r(a_1b_1+a_2b_2+\cdots+a_nb_n)=(ra_1)b_1+(ra_2)b_2+\cdots+(ra_n)b_n$. Since $ra_k\in I$ by Theorem 6.1 and $b_k\in J$, $rp\in IJ$. Similarly $pr\in IJ$ since $pr=(a_1b_1+a_2b_2+\cdots+a_nb_n)r=a_1(b_1r)+a_2(b_2r)+\cdots+a_n(b_nr)$. Therefore $IJ$ is an ideal by Theorem 6.1.
            \end{enumerate}
      \item Let $R$ be an integral domain and $a,b\in R$. Show that $(a)=(b)$ if and only if $a=bu$ for some unit $u\in R$.\\
            \textbf{Answer}:
            \begin{itemize}
                  \item [$\Rightarrow$:] Since $(a)=(b)$, we can take $ra=rb\cdot 1_R$ for every element of $(a)$ and $(b)$ ($1_R$ always exists since $R$ is an integral domain), then we have $a=bu$ with $u=1_R$.
                  \item [$\Leftarrow$:] Since $a=bu$, every element of $(b)$ is a multiple of $a$ in $R$. Then by definition of principal ideal (Theorem 6.2) $(a)=(b)$.
            \end{itemize}
      \item Let $R$ be a commutative ring with $1_R\neq 0_R$, whose only ideals are $(0)$ and $R$. Prove that $R$ is a field.\\
            \textbf{Answer}: Take $a\in R$ where $a\neq 0$, then since $r\neq 0\implies (a)\neq (0)$ we have $(a)=R$. Therefore $1_R\in (a)$ and by definition of ideal there exists some $r\in R$ such that $ar=1_R$. Therefore we can always take $x=r$ as the solution to $ax=1_R$, so $R$ is a field.
      \item Let $I$ and $K$ be ideals in a ring $R$, with $K\subseteq I$. Prove that $I/K=\{a+K\mid a\in I\}$ is an ideal in the quotient ring $R/K$.\\
            \textbf{Answer}: Take $a+K,b+K\in I/K$, we have $(a+K)-(b+K)=(a-b)+K\in I/K$. Now take $r+K\in R/K$, we have $(a+K)(r+K)=ar+K\in I/K$ and $(r+K)(a+K)=ra+K\in I/K$. Therefore $I/K$ is an ideal by Theorem 6.1.
      \item \textit{\underline{The Third Isomorphism Theorem}}: Let $R$ be a ring, and let $I,K\subseteq R$ be ideals with $K\subseteq I$. By problem 5, $I/K$ is an ideal of $R/K$. Prove that $(R/K)/(I/K)\approxeq R/I$.\\
            \textbf{Answer}: Take $f:R/K\rightarrow R/I,f(r+K)=r+I$. We have \[f((a+K)+(b+K)=f((a+b)+K)=(a+b)+I=(a+I)+(b+I)=f(a+K)+f(b+K)\] and \[f((a+K)(b+K))=f(ab+K)=ab+I=(a+I)(b+I)=f(a+K)f(b+K).\] Therefore $f$ is a homomorphism. It is also surjective since for every $r+I\in R/I$ we can take $r+K\in R/K$ such that $f(r+K)=r+I$. Note that the kernel of $f$ is $I/K$ since $f(r+K)=I\implies r\in I$ and $r+K\in I/K\implies r\in I\implies f(r+K)=r+I=I$. Then by the First Isomorphism Theorem $(R/K)/(I/K)\approxeq R/I$.
      \item \textit{\underline{The Chinese Remainder Theorem}}: Let $m,n\in\mathbb{Z}$ be two \textit{relatively prime} positive intgers.
            \begin{enumerate}
                  \item Show that the function $f:\mathbb{Z}\rightarrow(\mathbb{Z}/m\mathbb{Z})\cross(\mathbb{Z}/n\mathbb{Z})$ defined by $f(x)=([x]_m,[x]_n)$ is a homomorphism.\\
                        \textbf{Answer}: We have \[f(a+b)=([a+b]_m,[a+b]_n)=([a]_m,[a]_n)+([b]_m,[b]_n)=f(a)+f(b)\] and \[f(ab)=([ab]_m,[ab]_n)=([a]_m,[a]_n)([b]_m,[b]_n)=f(a)f(b).\] Therefore $f$ is a homomorphism.
                  \item Show that $\ker f=mn\mathbb{Z}$.\\
                        \textbf{Answer}: Take $mnk\in mn\mathbb{Z}$, we have $f(mnk)=([mnk]_m,[mnk]_n)=([0]_m,[0]_n)$ since $mnk\equiv 0$ (mod $m$) and $mnk\equiv 0$ (mod $n$), so $mn\mathbb{Z}\subseteq\ker f$. Now take $a\in\mathbb{Z}$ such that $f(a)=([0]_m,[0]_n)$, we must have $a\equiv 0$ (mod $m$) and $a\equiv 0$ (mod $n$), i.e. $m|a$ and $n|a$. Then $mn|a$, so $a\in mn\mathbb{Z}$ and $\ker f\subseteq mn\mathbb{Z}$. Therefore $\ker f=mn\mathbb{Z}$.
                  \item Use the first isomorphism theorem to show that $\mathbb{Z}/mn\mathbb{Z}\approxeq(\mathbb{Z}/m\mathbb{Z})\cross(\mathbb{Z}/n\mathbb{Z})$.\\
                        \textbf{Answer}: $f$ is surjective since for every $([x]_m,[x]_n)\in(\mathbb{Z}/m\mathbb{Z})\cross(\mathbb{Z}/n\mathbb{Z})$ we always have $x\in\mathbb{Z}$. By part (a) $f$ is a homomorphism, so $f$ is a surjective homomorphism of rings. By part (b) $\ker f=mn\mathbb{Z}$, so by the First Isomorphism Theorem the quotient ring $\mathbb{Z}/mn\mathbb{Z}$ is isomorphic to $(\mathbb{Z}/m\mathbb{Z})\cross(\mathbb{Z}/n\mathbb{Z})$.
                  \item Use part (c) to prove the Chinese Remainder Theorem: If $m,n\in\mathbb{Z}$ are relatively prime, and $a,b\in\mathbb{Z}$ are any integers, then there is a \textit{unique} congruence class $[x]\in\mathbb{Z}/mn\mathbb{Z}$ satisfying the system of congruences \[x\equiv a(\text{mod }m)\]\[x\equiv b(\text{mod }n).\] (in other words, there is some integer $x$ satisfying both of these congruences, and if $x$ and $x'$ both satisfy these congruences then $x\equiv x'$ (mod $mn$)).\\
                        \textbf{Answer}: By part (c) we have $\mathbb{Z}/mn\mathbb{Z}\approxeq(\mathbb{Z}/m\mathbb{Z})\cross(\mathbb{Z}/n\mathbb{Z})$, so there exists an $x\in\mathbb{Z}/mn\mathbb{Z}$ such that $x\equiv a$ (mod $m$) and $x\equiv a$ (mod $n$). In addition, since we have an isomorphism such $x$ is unique, i.e. $x\equiv x'$ if $x,x'$ both satisfy the congruences.
            \end{enumerate}
      \item Let $f:R\rightarrow S$ be a surjective homomorphism of commutative rings. If $J$ is a prime ideal in $S$ and $I=\{r\in R\mid f(r)\in J\}$, prove that $I$ is a prime ideal in $R$.\\
            \textbf{Answer}: Since $J$ is a prime ideal in $S$, $J\neq S$, so $S-J$ is not empty. Then for any $r\in R$ and $s\in S-J$ such that $f(r)=s$, we have $r\notin I$. So $R-I$ is also not empty and therefore $I\neq R$. Now, for $ab\in I$, we have $f(ab)=f(a)f(b)\in J$. By definition of prime ideal we must have either $f(a)\in J$ or $f(b)\in J$, therefore we must have either $a\in I$ or $b\in I$, so by definition of prime ideal $I$ is a prime ideal.
\end{enumerate}
\end{document}