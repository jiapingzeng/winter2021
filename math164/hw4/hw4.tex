\documentclass{article}
\usepackage[margin=1in]{geometry}
\usepackage{enumitem}
\usepackage{setspace}
\usepackage{amsmath}
\usepackage{amssymb}
\usepackage{physics}
\usepackage{relsize}
\usepackage{graphicx}

\title{Math 164 Homework 3}
\date{2/15/2021}
\author{Jiaping Zeng}

\begin{document}
\setstretch{1.35}
\maketitle

\begin{enumerate}
      \item Show each of the following set is convex.
            \begin{enumerate}
                  \item Hyperplane: $\{x:a^Tx=b\}$ with $a\neq 0\in\mathbb{R}^n$ and $b\in\mathbb{R}$.\\
                        \textbf{Answer}: Take $x,y\in\Omega\implies a^Tx=a^Ty=b$, for $\alpha\in[0,1]$, we have $\alpha b+(1-\alpha)b=b\implies\alpha a^Tx+(1-\alpha) a^Ty=b\implies a^T(\alpha x+(1-\alpha)y)=b$, so $\alpha x+(1-\alpha)y\in\Omega$ and therefore $\Omega$ is a convex set.
                  \item Halfspace: $\{x:a^Tx\leq b\}$ with $a\neq 0\in\mathbb{R}^n$ and $b\in\mathbb{R}$.\\
                        \textbf{Answer}: Take $x,y\in\Omega\implies a^Tx\leq b,b^Tx\leq b$, for $\alpha\in[0,1]$, we have $\alpha a^Tx\leq\alpha b,(1-\alpha)a^Ty\leq (1-\alpha)b\implies\alpha a^Tx+(1-\alpha)a^Ty\leq\alpha b+(1-\alpha)b\implies a^T(\alpha x+(1-\alpha)y)\leq b$, so $\alpha x+(1-\alpha)y\in\Omega$ and therefore $\Omega$ is a convex set.
                  \item Norm Ball: $\{x:\norm{x-x_c}\leq r\}$ with $>0$ and $x_c\in\mathbb{R}^n$.\\
                        \textbf{Answer}: Take $x,y\in\Omega\implies\norm{x-x_c}\leq r,\norm{y-y_c}\leq r$, for $\alpha\in[0,1]$, we have $((1-\alpha)\norm{x-x_c}+\alpha\norm{y-y_c})$
                  \item Polyhedron: $\{x:Ax\leq b,Cx=d\}$ with $A^{m\cross n},C\in\mathbb{R}^{p\cross n}$ and $b\in\mathbb{R}^m,d\in\mathbb{R}^p$.\\
                        \textbf{Answer}:
                  \item Nonnegative Orthant: $\mathbb{R}_+^n=\{x:x\geq 0\}$.\\
                        \textbf{Answer}:
                  \item Positive semidefinite cone: $S_+^n=\{X\in\mathbb{R}^{n\cross n}:X=X^T,X\succeq 0\}$\\
                        \textbf{Answer}:
            \end{enumerate}
      \item For each of the following functions, determine whether it is convex or concave or neither or both. Please explain the rationale for your answer.
            \begin{enumerate}
                  \item $f(x_1,x_2)=x_1^2+x_2^4$.\\
                        \textbf{Answer}:
                  \item $f(x_1,x_2)=e^{ax_1}+e^{bx_2}$ with $a,b\in\mathbb{R}$.\\
                        \textbf{Answer}:
                  \item $f(x_1,x_2)=x_1\log(x_1)+x_2\log(x_2)$ on $\mathbb{R}_{++}^2$.\\
                        \textbf{Answer}:
                  \item $f(x_1,x_2)=x_1x_2$ on $\mathbb{R}_{++}^2$.\\
                        \textbf{Answer}:
                  \item $f(x_1,x_2)=1/(x_1x_2)$ on $\mathbb{R}_{++}^2$.\\
                        \textbf{Answer}:
                  \item $f(x_1,x_2)=x_1^2/x_2$ on $\mathbb{R}\cross\mathbb{R}_{++}$.\\
                        \textbf{Answer}:
            \end{enumerate}
      \item Suppose $f:\mathbb{R}^n\rightarrow\mathbb{R}$ is a convex function. Show that the set of the global  minimizers of $\min_xf(x)$ is a convex.\\
            \textbf{Answer}:
      \item Suppose $f:\mathbb{R}^n\rightarrow\mathbb{R}$ is a continuously differentiable and convex function. Show that $\nabla f(x^*)=0$ if and only if $x^*$ is a global minimizer of $\min_xf(x)$.\\
            \textbf{Answer}:
\end{enumerate}
\end{document}